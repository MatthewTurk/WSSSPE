\documentclass[11pt, oneside]{amsart}

\usepackage{amsmath}
\usepackage{amssymb}

\usepackage{color}
\usepackage{dcolumn}
\usepackage{float}
\usepackage{graphicx}
\usepackage[latin9]{inputenc}
\usepackage{multirow}
\usepackage{rotating}
\usepackage{subfigure}
\usepackage{psfrag}
\usepackage{tabularx}
\usepackage[hyphens]{url}
\usepackage{wrapfig}

%\setcounter{secnumdepth}{3}
%\setcounter{tocdepth}{3}


\usepackage[bookmarks, bookmarksopen, bookmarksnumbered]{hyperref}
\usepackage[all]{hypcap}
\urlstyle{rm}

\definecolor{orange}{rgb}{1.0,0.3,0.0}
\definecolor{violet}{rgb}{0.75,0,1}
\definecolor{darkgreen}{rgb}{0,0.6,0}
\definecolor{cyan}{rgb}{0.2,0.7,0.7}
\definecolor{blueish}{rgb}{0.2,0.2,0.8}

\newcommand{\todo}[1]{{\color{blue}$\blacksquare$~\textsf{[TODO: #1]}}}
\newcommand{\katznote}[1]{ {\textcolor{magenta}    { ***Dan:      #1 }}}
\newcommand{\gabnote}[1]{ {\textcolor{cyan}    { ***Gabrielle:     #1 }}}
\newcommand{\nchnote}[1]{  {\textcolor{orange}      { ***Neil: #1 }}}
\newcommand{\manishnote}[1]{  {\textcolor{violet}     { ***Manish: #1 }}}
\newcommand{\davidnote}[1]{  {\textcolor{darkgreen}      { ***David: #1 }}}
\newcommand{\note}[1]{ {\textcolor{red}    { #1 }}}


% Don't use tt font for urls
\urlstyle{rm}

% 15 characters / 2.5 cm => 100 characters / line
% Using 11 pt => 94 characters / line
\setlength{\paperwidth}{216 mm}
% 6 lines / 2.5 cm => 55 lines / page
% Using 11pt => 48 lines / pages
\setlength{\paperheight}{279 mm}
\usepackage[top=2.5cm, bottom=2.5cm, left=2.5cm, right=2.5cm]{geometry}
% You can use a baselinestretch of down to 0.9
\renewcommand{\baselinestretch}{0.96}

\sloppypar

\begin{document}

\title[]{Overview and Summary of the First Workshop on Sustainable Software for Science: Practice and Experiences (WSSSPE)}

\author{Authors TBD}

%\author{Daniel S. Katz$^{(1)}$, Gabrielle Allen$^{(2)}$, Neil Chue Hong$^{(3)}$, \\
%Manish Parashar$^{(4)}$, and David Proctor$^{(1)}$ }
%
%
%\thanks{{}$^{(1)}$ National Science Foundation, Arlington, VA, USA}
%
%\thanks{{}$^{(2)}$ Skolkovo Institute of Science and Technology, Moscow, Russian Federation}
%
%\thanks{{}$^{(3)}$ Software Sustainability Institute, University of Edinburgh, Edinburgh, UK}
%  
%\thanks{{}$^{(4)}$ Rutgers Discovery Informatics Institute, Rutgers University, New Brunswick, NJ, USA}

    
\begin{abstract}
This technical report discusses the First Workshop on Sustainable Software for Science: Practice and Experiences (WSSSPE). ...
\end{abstract}


\maketitle

\note{papers and presentations should be references - URLs perhaps can just be footnotes for now, but may want to think about this again.}

People who have volunteered to work on this report: (with * for actual contributors - \note{if you contribute, add a * by your name})

The workshop organizers:
\begin{itemize}
\item * Daniel S. Katz $\langle$\url{dkatz@nsf.gov}$\rangle$
\item Gabrielle Allen $\langle$\url{gdallen@illinois.edu}$\rangle$
\item Neil Chue Hong $\langle$\url{N.ChueHong@software.ac.uk}$\rangle$
\item Manish Parashar $\langle$\url{parashar@rutgers.edu}$\rangle$
\item David Proctor $\langle$\url{dproctor@nsf.gov}$\rangle$
\end{itemize}


Additional volunteers:
\begin{itemize}
\item Chris Mattmann $\langle$\url{chris.a.mattmann@jpl.nasa.gov}$\rangle$
\item * Ketan Maheshwari $\langle$\url{ketancmaheshwari@gmail.com}$\rangle$
\item Marlon Pierce $\langle$\url{marpierc@iu.edu}$\rangle$
\item Colin Venters $\langle$\url{C.Venters@hud.ac.uk}$\rangle$
\item Suresh Marru $\langle$\url{smarru@iu.edu}$\rangle$
\item Lynn Zentner $\langle$\url{lzentner@purdue.edu}$\rangle$
\item Anne Elster $\langle$\url{anne.elster@gmail.com}$\rangle$
\item Shel Swenson $\langle$\url{mdswenso@usc.edu}$\rangle$
\item Andy Ray Terrel $\langle$\url{andy.terrel@gmail.com}$\rangle$
\item Abani Patra $\langle$\url{abani.patra@gmail.com}$\rangle$
\item Nancy Wilkins-Diehr $\langle$\url{wilkinsn@sdsc.edu}$\rangle$
\item James Spencer $\langle$\url{j.spencer@imperial.ac.uk}$\rangle$
\item Frank Loeffler $\langle$\url{knarf@cct.lsu.edu}$\rangle$
\item * James Hetherington $\langle$\url{j.hetherington@ucl.ac.uk}$\rangle$
\item * Sou-Cheng (Terrya) Choi $\langle$\url{sou.cheng.terrya.choi@gmail.com}$\rangle$
\item James Howison $\langle$\url{james@howison.name}$\rangle$
\item * Bruce Berriman $\langle$\url{gbb@ipac.caltech.edu}$\rangle$
\item Hilmar Lapp $\langle$\url{hlapp@nescent.org}$\rangle$
\item * Marcus D. Hanwell $\langle$\url{marcus.hanwell@kitware.com}$\rangle$
\item Lucas Nussbaum $\langle$\url{lucas.nussbaum@loria.fr}$\rangle$
\item Matthew Turk $\langle$\url{matthewturk@gmail.com}$\rangle$
\end{itemize}

The original document is \url{https://docs.google.com/document/d/1eVfioGNlihXG_1Y8BgdCI6tXZKrybZgz5XuQHjT1oKU/edit?pli=1#} (but can no longer be edited).
Note that the original document has comments in addition to text.


\pagebreak

\section*{Executive Summary}

\todo{1 page summary goes here}

\pagebreak

\section{Introduction}

The First Workshop on Sustainable Software for Science: Practice and Experiences (WSSSPE1)\footnote{\url{http://wssspe.researchcomputing.org.uk/WSSSPE1}} was held on Sunday, 17 November 2013, in conjunction with the 2013 International Conference for High Performance Computing, Networking, Storage and Analysis (SC13)\footnote{\url{http://sc13.supercomputing.org}}.

Because progress in scientific research is dependent on the quality and accessibility
of software at all levels, it is now critical to address many challenges related to the
development, deployment, and maintenance of reusable software. In addition, it is
essential that scientists, researchers, and students are able to learn and adopt
software-related skills and methodologies. Established researchers are already
acquiring some of these skills, and in particular a specialized class of software
developers is emerging in academic environments who are an integral and embedded
part of successful research teams. The WSSSPE workshop was intended provide a
forum for discussion of the challenges, including both positions and experiences.
The short papers and discussion have been archived to provide a basis for continued
discussion, and the workshop has fed into the collaborative writing of this journal publication.
Additional papers based on extended versions of workshop submissions are expected.
The level of interest in the workshop has led the organizers, working with some of the submitters
and some of the attendees, to plan two additional workshops (\todo{details}) and to turn
the workshop website into a community website that can be used as a focus for further
discussion and progress.

\section{Workshop Process}

WSSSPE1 was organized via peer-review and organization of papers submitted in response to a call for papers.  The call for paper was created by the organizers and peer-review committee.

The workshop call for paper included:

\begin{quote}
In practice, scientific software activities are part of an ecosystem where key roles are held by 
developers, users, and funders.  All three groups supply resources to the ecosystem, as well as 
requirements that bound it.  Roughly following the example of NSF's Vision and Strategy for Software 
(\url{http://www.nsf.gov/publications/pub_summ.jsp?ods_key=nsf12113})~\cite{NSF_software_vision}, 
the ecosystem may be viewed as having challenges related to:

\begin{itemize}
\item the development process that leads to new software
\begin{itemize}
\item how fundamental research in computer science or science/engineering domains is turned  into reusable software
\item software created as a by-product of research
\item impact of computer science research on the development of scientific software
\end{itemize}
\item the support and maintenance of existing software, including software engineering
\begin{itemize}
\item governance, business, and sustainability models
\item the role of community software repositories, their operation and sustainability
\end{itemize}
\item the role of open source communities or industry
\item use of the software
\begin{itemize}
\item growing communities
\item reproducibility, transparency needs that may be unique to science
\end{itemize}
\item policy issues, such as
\begin{itemize}
\item measuring usage and impact
\item software credit, attribution, incentive, and reward
\item career paths for developers and institutional roles
\item issues related to multiple organizations and multiple countries, such as intellectual property, licensing, etc.
\item mechanisms and venues for publishing software, and the role of publishers
\end{itemize}
\item education and training
\end{itemize}

\end{quote}

Based on the goal of encouraging a wide range of submissions from those involved in software practice, ranging from initial thoughts and partial studies to mature deployments, the organizers wanted to make submission as easy as possible.  The call for papers stated:

\begin{quote}
We invite short (4-page) position/experience reports that will be used to organize panel and 
discussion sessions.  These papers will be archived by a third-party service, and provided DOIs. We 
encourage submitters to license their papers under a Creative Commons license that encourages 
sharing and remixing, as we will combine ideas (with attribution) into the outcomes of the workshop.  
An interactive site will be created to link these papers and the workshop discussion, with options for 
later comments and contributions.  Contributions will be peer-reviewed for relevance and originality 
before the links are added to the workshop site; contributions will also be used to determine 
discussion topics and panelists.  We will also plan one or more papers to be collaboratively 
developed by the contributors, based on the panels and discussions.
\end{quote}

58 submissions were received, and almost all submitters used either 
arXiv\footnote{\url{http://arxiv.org}} or 
figshare\footnote{\url{http://figshare.com}} to self-publish their papers.  

A peer review process followed the submissions, where the 58 papers received 181 reviews, an 
average of 3.12 reviews per paper.  Reviews were completed using a Google form, which allowed 
reviewers to provide scores on relevance and comments to the organizers, which were used to 
decide which papers to associate with the workshop, and comments to the authors, which were 
provided back to the authors to allow them to improve their papers.

The organizers decided to list 54 of the papers as significantly contributing to the workshop, a very 
high acceptance rate, but one that is reasonable, given the goal of broad participation and the fact 
that the reports were already self-published.  The papers were also grouped into 3 areas, each of 
which was associated with a panel and discussion at the workshop.


\section{Agenda}

The workshop consisted of 2 keynote presentations and three panel/discussion sessions.  The panels 
were organized based on a classification of the workshop submissions into three categories, following 
the themes of the call for papers as modified by the areas of the submissions.  Representatives from 
each submission categories were appointed as panelists, to discuss the papers in that category.


\section{Keynotes}

\todo{section should include what was presented and what was discussed}

\subsection{A recipe for sustainable software, Philip E. Bourne.}

The first keynote\footnote{\url{http://wssspe.researchcomputing.org.uk/keynotes/\#bourne}}
was delivered by Philip E. Bourne of UCSD. The slides
are online.\footnote{\url{www.slideshare.net/pebourne/a-recipe-for-sustainable-software}}

Philip is a basic biomedical scientist but has formed 4 software
companies. He also helped found PLOS and PDB\footnote{\url{http://www.rcsb.org/pdb/home/home.do}}.

He emphasized that Sustainability for software ``does not just mean more
money from Government.''

He noted that creating effective illustrations is a key function of
software in the biomedical sciences: ``Automating Conceptualization''
(Visualizing molecular structures is a problem that has now been solved,
he is working on 3-d visualizations of the contents of cells.)

He has looked into reproducibility in computational biology, and has
concluded that ``I have proved I cannot reproduce research from my own
lab'':
(Computational
Biology Resources Lack Persistence And Usability, Veretnik, Fink and
Bourne.)\footnote{\url{http://www.ploscompbiol.org/article/info\%3Adoi\%2F10.1371\%2Fjournal.pcbi.1000136}} \todo{change this to a cite}

He noted that in places, development in science has improved thanks to
open source and Github, but for the most part remains arcane. He argues
that there's a lot we can learn from the app-store model. He
mentioned BioJava\footnote{\url{http://biojava.org/wiki/Main_Page}} and
OpenScienceDataCloud \footnote{\url{https://www.opensciencedatacloud.org/}} as
distribution channels.

He noted a common evolutionary pathway for computational biology
projects, from data archive to analytics platform to educational use. He
suggests that use for outreach might be the final step.

He has problems with staff retention with respect to private organizations which reward 
those combining research and software expertise (the ``Google Bus''), however,
he's a strong fan of private sector sustainability sources and PPP. He
notes that making a successful business from scientific software alone
is very rare: founders overvalue while customers undervalue. He noted
that to last, an open source project needs a minimal funding requirement
even with a vibrant community -- good will only goes so far if you're
being paid to do something else.

He talked about grant schemes of relevance in the U.S., particularly with regard to technology transfer.\footnote{\url{http://www.nsf.gov/pubs/policydocs/pappguide/nsf09_1/gpg_2.jsp\#IID2}} \footnote{\url{http://www.nsf.gov/eng/iip/sbir/}} \todo{change at least first to cite}

He had problems with selling research software: his technology transfer
office wanted wanted huge IP reachthrough, whereby they would get a
share of profits from drugs developed by pharma companies who use the
software. He was aware this was unrealistic but the technology transfer
office insisted for a
while. He wants to push a much more one-click approach to purchasing
university-written software. 

He then presented the usual arguments on directly valuing software as a
research output alongside papers. Readers will be familiar with this
debate. An interesting reference here is ``What
does peer review mean when applied to computer code?, Rosemary Dickin.''
\footnote{\url{http://blogs.plos.org/biologue/2013/08/08/what-does-peer-review-mean-when-applied-to-computer-code/}} \todo{change to cite}

He then expanded on the idea of the \textbf{digital enterprise}, where
IT underpins the whole of organizational activities. He contends that
universities are way behind the curve on this. In particular, he
highlights the separation of research, teaching and administration into
silos without a common IT framework as a blocker to many useful
organizational innovations: ``University 2.0 is yet to happen.'' He
spoke of a circumstance where someone had used an algorithm developed
for computational biology in marketing.

The role of an institution is important in this space. He argued that
funders can help train institutions not just individuals.

He concluded with a reference to his paper
\footnote{\url{http://www.ploscompbiol.org/article/info\%3Adoi\%2F10.1371\%2Fjournal.pcbi.1002001}}``Ten
simple rules for getting ahead as a computational biologist in
academia'' \todo{change to cite} and argued that computational scientists ``have a
responsibility to convince their institutions and communities that
software is scholarship,'' a point with which we strongly agree.

\subsection{Scientific Software and the Open Collaborative Web}

\todo{change references to papers or presentations to cites}

The second keynote\footnote{\url{http://is.gd/wssspe}} was by
 Arfon Smith\footnote{\url{http://arfon.org/}} of Github.

His talk used an example from data reduction in Astronomy, where one
attempts to remove interfering effects from the device. In this field, a
``bad pixel mask'' is recalculated by each researcher, despite being
persistent over time. He estimated 13 wasted person-years of waste.

He asked, of this practice, ``why didn't we know better?'' and argued
this was because we're taught to focus on immediate research outcomes
and not on continuously improving and building on tools for research. He
then asked, later in our working lives, when we do know better, why
don't we act any different, and argued this was due to incentivization:
only the immediate products of research, not the software, are valued.

He referenced Victoria Stodden's talk at OKCon\footnote{\url{http://okcon.org/main-speakers/victoria-stodden/}},\footnote{\url{http://www.stanford.edu/~vcs/talks/OKcon2013-Sept172013-STODDEN.pdf}}
which he said argued these points well.

He cited the EMCEE\footnote{\url{https://github.com/dfm/emcee}} Markov Chain
Monte Carlo code by Dan Forman Mackey\footnote{\url{http://dan.iel.fm/}} as a
good example of a scientific project using GitHub. Amongst other
examples of good scientific projects on GitHub he used an example from Joe Zuntz\footnote{\url{http://www.jb.man.ac.uk/~zuntz/Joe_Zuntz/Welcome.html}}.

C. Titus Brown\footnote{\url{http://ged.msu.edu/}} argues that with regard to
reusable software, ``we should just start doing it''. Arfon notes that
documentation should be ``treated as a first class entity.''

He noted that some software fields have excellent cultures of module
reuse, such as rubygems\footnote{\url{http://rubygems.org/}},
PyPI\footnote{\url{https://pypi.python.org/pypi}}, and CPAN\footnote{\url{http://www.cpan.org/}}, where there is effectively
ultra-low-friction collaboration through these repositories. This has
not happened in highly numerical, compiled language scientific software.

He argues that the GitHub Pull Request code review mechanism facilitates
such collaboration, by allowing one to code first, and seek review and a
merge back into the trunk later.

Fernando Perez says that ``Open Source
is reproducible by necessity'' - since reproducibility is a prerequisite
for remote collaboration.

Arfon therefore defines GitHub as part of the next stage in development
of the web, following on from the ``social web'' of Facebook et.al.:
``The Collaborative Web.''

\section{Developing and Supporting Software}

\todo{section should include what was presented and what was discussed}

\subsection{Papers}

\subsubsection*{Development Experiences}

\begin{itemize}

\item Mark C. Miller, Lori Diachin, Satish Balay, Lois Curfman McInnes, Barry Smith. Package Management Practices Essential for Interoperability: Lessons Learned and Strategies Developed for FASTMath \cite{Miller_WSSSPE}

\item Karl W. Broman, Thirteen years of R/qtl: Just barely sustainable \cite{Broman_WSSSPE}

\item Charles R. Ferenbaugh, Experiments in Sustainable Software Practices for Future Architectures \cite{Ferenbaugh_WSSSPE}

\item Eric G Stephan, Todd O Elsethagen, Kerstin Kleese van Dam, Laura Riihimaki. What Comes First, the OWL or the Bean? \cite{Stephan_WSSSPE}

\item Derek R. Gaston, John Peterson, Cody J. Permann, David Andrs, Andrew E. Slaughter, Jason M. Miller, Continuous Integration for Concurrent Computational Framework and Application Development \cite{Gaston_WSSSPE}

\item Anshu Dubey, B. Van Straalen. Experiences from Software Engineering of Large Scale AMR Multiphysics Code Frameworks \cite{Dubey_WSSSPE}

\item Markus Blatt. DUNE as an Example of Sustainable Open Source Scientific Software Development \cite{Blatt_WSSSPE}

\item David Koop, Juliana Freiere, Cl\'{a}udio T. Silva, Enabling Reproducible Science with VisTrails~\cite{Koop_WSSSPE}

\item Sean Ahern, Eric Brugger, Brad Whitlock, Jeremy S. Meredith, Kathleen Biagas, Mark C. Miller, Hank Childs, VisIt: Experiences with Sustainable Software \cite{Ahern_WSSSPE}

\item Sou-Cheng (Terrya) Choi. MINRES-QLP Pack and Reliable Reproducible Research via Staunch Scientific Software \cite{Choi_WSSSPE}

\item Michael Crusoe, C. Titus Brown. Walking the talk: adopting and adapting sustainable scientific software development processes in a small biology lab \cite{Crusoe_WSSSPE}

\item Dhabaleswar K. Panda, Karen Tomko, Karl Schulz, Amitava Majumdar. The MVAPICH Project: Evolution and Sustainability of an Open Source Production Quality MPI Library for HPC \cite{Panda_WSSSPE}

\item Eric M. Heien, Todd L. Miller, Becky Gietzel, Louise H. Kellogg. Experiences with Automated Build and Test for Geodynamics Simulation Codes \cite{Heien_WSSSPE}

\end{itemize}

\subsubsection*{Deployment, Support, and Maintenance of Existing Software}

\begin{itemize}

\item Henri Casanova, Arnaud Giersch, Arnaud Legrand, Martin Quinson, Fr\'{e}d\'{e}ric Suter. SimGrid: a Sustained Effort for the Versatile Simulation of Large Scale Distributed Systems~\cite{Casanova_WSSSPE}

\item Erik Trainer, Chalalai Chaihirunkarn, James Herbsleb. The Big Effects of Short-term Efforts: A Catalyst for Community Engagement in Scientific Software \cite{Trainer_WSSSPE}

\item Jeremy Cohen, Chris Cantwell, Neil Chue Hong, David Moxey, Malcolm Illingworth, Andrew Turner, John Darlington, Spencer Sherwin. Simplifying the Development, Use and Sustainability of HPC Software \cite{Cohen_WSSSPE}

\item Jaroslaw Slawinski, Vaidy Sunderam. Towards Semi-Automatic Deployment of Scientific and Engineering Applications \cite{Slawinski_WSSSPE}

\end{itemize}

\subsubsection*{Best Practices, Challenges, and Recommendations}

\begin{itemize}

\item Andreas Prli\'{c}, James B. Procter. Ten Simple Rules for the Open Development of Scientific Software \cite{Prlic_WSSSPE}

\item Anshu Dubey, S. Brandt, R. Brower, M. Giles, P. Hovland, D. Q. Lamb, F. L\:{o}ffler, B. Norris, B. O'Shea, C. Rebbi, M. Snir, R. Thakur, Software Abstractions and Methodologies for HPC Simulation Codes on Future Architectures \cite{Dubey2_WSSSPE}

\item Jeffrey Carver, George K. Thiruvathukal. Software Engineering Need Not Be Difficult \cite{Carver_WSSSPE}

\item Craig A. Stewart, Julie Wernert, Eric A. Wernert, William K. Barnett, Von Welch. Initial Findings from a Study of Best Practices and Models for Cyberinfrastructure Software Sustainability \cite{Stewart_WSSSPE}

\item Jed Brown, Matthew Knepley, Barry Smith. Run-time extensibility: anything less is unsustainable \cite{Brown_WSSSPE}

\item Shel Swenson, Yogesh Simmhan, Viktor Prasanna, Manish Parashar, Jason Riedy, David Bader, Richard Vuduc. Sustainable Software Development for Next-Gen Sequencing (NGS) Bioinformatics on Emerging Platforms \cite{Swenson_WSSSPE}
 
\end{itemize}


\subsection{Research or Reuse?}

Discussion of software produced as a by-product versus software developed for reuse. How does this change the project, who develops the code, growth beyond 1--2 person projects to larger projects with diverse set of consumers and contributors. Modular and extensible code versus working for current research problem---tackled by same people, or a collaborative team?

\subsection{Defining Sustainability for Scientific Software}

What is sustainable? What are the best practices, can workshops fund meetings to help define and improve best practices in these areas? Software plans from the funding agencies? \todo{mention as cross-cutting, since this came up in other parts of the workshop as well?}


\subsection{Training Scientists to Develop and Support Software}

Discussion of the evolving role of scientists, and/or others that fill these roles.

\subsection{Software Process, Code Review, Automation, Reproducibility}

There are a lot of tools out there, but few are currently used. Look at what some projects have found successful, and how to automate as much as possible to reduce additional overhead.

\subsection{Software and Data Licensing}

Its impact on how and where research products are used, who can collaborate and what patterns have worked/not worked in existing communities.

\subsection{Funding, Sustainability Beyond the First Grant/Institution}

How initial software development is funded, moving to follow up projects, maintenance, community growth, collaborating with other institutions, labs, industry, internationally. Contract relationship versus collaborative development between scientists and software developers.

\subsection{Training Others to use Software}

Who creates training materials, how are they distributed, integration into courses when projects see very wide application in research.

\section{Policy}

\todo{section should include what was presented and what was discussed}

\subsection{Papers}

\subsubsection*{Modeling Sustainability}

\begin{itemize}

\item Coral Calero, M. Angeles Moraga, Manuel F. Bertoa. Towards a Software Product Sustainability Model \cite{Calero_WSSSPE}

\item Colin C. Venters, Lydia Lau, Michael K. Griffiths, Violeta Holmes, Rupert R. Ward, Jie Xu. The Blind Men and the Elephant: Towards a Software Sustainability Architectural Evaluation Framework \cite{Venters_WSSSPE}

\item Marlon Pierce, Suresh Marru, Chris Mattmann. Sustainable Cyberinfrastructure Software Through Open Governance \cite{Pierce_WSSSPE}

\item Daniel S. Katz, David Proctor. A Framework for Discussing e-Research Infrastructure Sustainability \cite{Katz_WSSSPE}

\item Christopher Lenhardt, Stanley Ahalt, Brian Blanton, Laura Christopherson, Ray Idaszak. Data Management Lifecycle and Software Lifecycle Management in the Context of Conducting Science \cite{Lenhardt_WSSSPE}

\item Nicholas Weber, Andrea Thomer, Michael Twidale. Niche Modeling: Ecological Metaphors for Sustainable Software in Science \cite{Weber_WSSSPE}

\end{itemize}

\subsubsection*{Credit, Citation, Impact}

\begin{itemize}

\item Matthew Knepley, Jed Brown, Lois Curfman McInnes, Barry Smith. Accurately Citing Software and Algorithms Used in Publications \cite{Knepley_WSSSPE}

\item Jason Priem, Heather Piwowar. Toward a comprehensive impact report for every software project \cite{Priem_WSSSPE}

\item Daniel S. Katz. Citation and Attribution of Digital Products: Social and Technological Concerns \cite{Katz2_WSSSPE}

\item Neil Chue Hong, Brian Hole, Samuel Moore. Software Papers: improving the reusability and sustainability of scientific software \cite{Chue_Hong_WSSSPE}

\end{itemize}

In addition, the following paper from another area were also discussed in this area.

\begin{itemize}

\item Frank L\"{o}ffler, Steven R. Brandt, Gabrielle Allen and Erik Schnetter. Cactus: Issues for Sustainable Simulation Software \cite{Loffler_WSSSPE}

\end{itemize}

\subsubsection*{Reproducibility}

\begin{itemize}

\item Victoria Stodden, Sheila Miguez. Best Practices for Computational Science: Software Infrastructure and Environments for Reproducible and Extensible Research \cite{Stodden_WSSSPE}

\end{itemize}

\subsubsection*{Implementing Policy}

\begin{itemize}

\item Randy Heiland, Betsy Thomas, Von Welch, Craig Jackson. Toward a Research Software Security Maturity Model \cite{Heiland_WSSSPE}

\item Brian Blanton, Chris Lenhardt, A User Perspective on Sustainable Scientific Software \cite{Blanton_WSSSPE}

\item Daisie Huang, Hilmar Lapp. Software Engineering as Instrumentation for the Long Tail of Scientific Software \cite{Huang_WSSSPE}

\item Rich Wolski, Chandra Krintz, Hiranya Jayathilaka, Stratos Dimopoulos, Alexander Pucher. Developing Systems for API Governance \cite{Wolski_WSSSPE}
 
\end{itemize}


\subsection{Career Tracks for Scientific Software Developers}

How to ensure software is sustainable by ensuring there are career paths for developers.

\section{Communities}

\todo{section should include what was presented and what was discussed}

\subsection{Papers}

\subsubsection*{Communities}

\begin{itemize}

\item L. Christopherson, R. Idaszak, S. Ahalt. Developing Scientific Software through the Open Community Engagement Process \cite{Christopherson_WSSSPE}

\item Reagan Moore. Extensible Generic Data Management Software \cite{Moore_WSSSPE}

\item Karen Cranston, Todd Vision, Brian O'Meara, Hilmar Lapp. A grassroots approach to software sustainability \cite{Cranston_WSSSPE}

\item J.-L. Vay, C. G. R. Geddes, A. Koniges, A. Friedman, D. P. Grote, D. L. Bruhwiler. White Paper on DOE-HEP Accelerator Modeling Science Activities \cite{Vay_WSSSPE}

\item Marlon Pierce, Suresh Marru, Mark A. Miller, Amit Majumdar, Borries Demeler. Science Gateway Operational Sustainability: Adopting a Platform-as-a-Service Approach \cite{Pierce2_WSSSPE}

\item Lynn Zentner, Michael Zentner, Victoria Farnsworth, Michael McLennan, Krishna Madhavan, and Gerhard Klimeck, nanoHUB.org: Experiences and Challenges in Software Sustainability for a Large Scientific Community \cite{Zentner_WSSSPE}

\item Andy Terrel. Sustaining the Python Scientific Software Community \cite{Terrel_WSSSPE}

\item Frank L\"{o}ffler, Steven R. Brandt, Gabrielle Allen and Erik Schnetter. Cactus: Issues for Sustainable Simulation Software \cite{Loffler_WSSSPE}

\item Ketan Maheshwari, David Kelly, Scott J. Krieder, Justin M. Wozniak, Daniel S. Katz, Mei Zhi-Gang, Mainak Mookherjee. Reusability in Science: From Initial User Engagement to Dissemination of Results \cite{Maheshwari_WSSSPE}

\item Nancy Wilkins-Diehr, Katherine Lawrence, Linda Hayden, Marlon Pierce, Suresh Marru, Michael McLennan, Michael Zentner, Rion Dooley, Dan Stanzione. Science Gateways and the Importance of Sustainability \cite{Wilkins-Diehr_WSSSPE}

\item Edmund Hart, Carl Boettiger, Karthik Ram, Scott Chamberlain. rOpenSci -- a collaborative effort to develop R-based tools for facilitating Open Science \cite{Hart_WSSSPE}

\end{itemize}

In addition, the following paper from another area was also discussed in this area.

\begin{itemize}

\item Marcus Hanwell, Amitha Perera, Wes Turner, Patrick O'Leary, Katie Osterdahl, Bill Hoffman, Will Schroeder. Sustainable Software Ecosystems for Open Science \cite{Hanwell_WSSSPE}

\end{itemize}

\subsubsection*{Industry \& Economic Models}

\begin{itemize}

\item Anne C. Elster. Software for Science: Some Personal Reflections \cite{Elster_WSSSPE}

\item Ian Foster, Vas Vasiliadis, Steven Tuecke. Software as a Service as a path to software sustainability \cite{Foster_WSSSPE}

\item Marcus Hanwell, Amitha Perera, Wes Turner, Patrick O'Leary, Katie Osterdahl, Bill Hoffman, Will Schroeder. Sustainable Software Ecosystems for Open Science \cite{Hanwell_WSSSPE}

\end{itemize}

In addition, the following papers from other areas were also discussed in this area.

\begin{itemize}

\item Brian Blanton, Chris Lenhardt, A User Perspective on Sustainable Scientific Software \cite{Blanton_WSSSPE}

\item Markus Blatt. DUNE as an Example of Sustainable Open Source Scientific Software Development \cite{Blatt_WSSSPE}

\item Dhabaleswar K. Panda, Karen Tomko, Karl Schulz, Amitava Majumdar. The MVAPICH Project: Evolution and Sustainability of an Open Source Production Quality MPI Library for HPC \cite{Panda_WSSSPE}

\item Andy Terrel. Sustaining the Python Scientific Software Community \cite{Terrel_WSSSPE}

\end{itemize}

\subsubsection*{Education \& Training}

\begin{itemize}

\item Ivan Girotto, Axel Kohlmeyer, David Grellscheid, Shawn T. Brown. Advanced Techniques for Scientific Programming and Collaborative Development of Open Source Software Packages at the International Centre for Theoretical Physics (ICTP) \cite{Girotto_WSSSPE}

\item Thomas Crawford. On the Development of Sustainable Software for Computational Chemistry \cite{Crawford_WSSSPE}

\end{itemize}

In addition, the following papers from other areas were also discussed in this area.

\begin{itemize}

\item Charles R. Ferenbaugh, Experiments in Sustainable Software Practices for Future Architectures \cite{Ferenbaugh_WSSSPE}

\item David Koop, Juliana Freiere, Cl\'{a}udio T. Silva, Enabling Reproducible Science with VisTrails~\cite{Koop_WSSSPE}

\item Sean Ahern, Eric Brugger, Brad Whitlock, Jeremy S. Meredith, Kathleen Biagas, Mark C. Miller, Hank Childs, VisIt: Experiences with Sustainable Software \cite{Ahern_WSSSPE}

\item Sou-Cheng (Terrya) Choi. MINRES-QLP Pack and Reliable Reproducible Research via Staunch Scientific Software \cite{Choi_WSSSPE}

\item Frank L\"{o}ffler, Steven R. Brandt, Gabrielle Allen and Erik Schnetter. Cactus: Issues for Sustainable Simulation Software \cite{Loffler_WSSSPE}

\item Erik Trainer, Chalalai Chaihirunkarn, James Herbsleb. The Big Effects of Short-term Efforts: A Catalyst for Community Engagement in Scientific Software \cite{Trainer_WSSSPE}

\end{itemize}

\subsection{What are communities?}
\subsection{Challenges of community}
include metrics for community success (e.g., more external contributors than internal).
\subsection{Admirable scientific software communities}
\begin{itemize}
\item example communities
\item their (quick) origin stories.
\end{itemize}
\subsection{Resources for learning about software communities}
\begin{itemize}
\item academic fields studying communities (and software communities)
\item courses on online communities
\item books
\item need for software carpentry module on organizing communities?
\end{itemize}

\section{Other Discussion}

\note{from the misc questions/answers/observations part of the google doc?}

\section{Cross-cutting Issues}

\subsection{Defining sustainability}

\subsection{Licensing}

\note{not sure if this was really cross-cutting...}


\section{Illustrative use-cases}

Perhaps discuss sustained software use-cases and issues surrounding them. Some examples could be mvapich, Galaxy, Swift, Cactus, etc.

\section{Conclusions}

\todo{pull the discussion together}

\todo{add some analysis of the deficiencies and difficulties that are present in different fields, and those that are common?}

\subsection{Recommendations or Lessons}

\note{if needed.}

\subsection{Follow up actions}

\note{or at least the discussion about them, and the current plans for future events.}

\note{conclusions from pre-workshop paper follows}

The WSSSPE workshop has begun an experiment in how we can collaboratively build a workshop 
agenda.  However, contributors also want to get credit for their participation in the process.  And the 
workshop organizers want to make sure that the workshop content and their efforts are recorded.  
Ideally, there would be a service that would be able to index the contributions to the workshop, 
serving the authors, the organizers, and the larger community.  But since there isn't such a service 
today, the workshop organizers are writing this initial report and making use of arXiv as a partial 
solution to provide a record of the workshop.

After the workshop, one or more additional papers will be created that will include the discussions at 
the workshop.  These papers will likely have many authors, and may be submitted to peer-reviewed 
journals.


\section*{Acknowledgments}

\appendix
\section{List of attendees}

The following is a partial list of attendees who were recorded on the Google doc that was being used for live note taking at the workshop:

Daniel S. Katz (@danielskatz)

Neil Chue Hong (@npch)

Gabrielle Allen

David Proctor (@djproctor)

Manish Parashar

Jay Alameda (alameda@illinois.edu)

David Andrs

Lorena A Barba (@LorenaABarba)

David Bernholdt (bernholdtde@ornl.gov)

Phil Bourne

Karl Broman (@kwbroman)

Maxine Brown, UIC (maxine@uic.edu)

David Bruhwiler   (bruhwiler@radiasoft.net    @radiasoft)

Bruno Bzeznik

Jeffrey Carver, University of Alabama (carver@cs.ua.edu)

John W. Cobb (cobbjw@ornl.gov, @johnwcobb)

Karen Cranston (@kcranstn, National Evolutionary Synthesis Center)

Rion Dooley

Anshu Dubey (adubey@lbl.gov)

Ian Foster, @ianfoster

Juliana Freire

Derek Gaston

Derek Groen (@whydoitweet on Twitter)

Edmund Hart (@distribecology)

James Hetherington @jamespjh

Simon Hettrick (s.hettrick@software.ac.uk)

Kenneth Hoste (@kehoste, Ghent University, Belgium)

James Howison, University of Texas at Austin (@jameshowison)

Daisie Huang

Kerk Kee (kerk.kee@gmail.com)

Arnaud Legrand

Paul Madden, NOAA / University of Colorado, maddenp@colorado.edu

Louis Maddox (@lmmddx)

Ketan Maheshwari (@km\_9 on Twitter)

Suresh Marru (smarru@iu.edu, @sureshmarru) 

Cezary Mazurek, mazurek@man.poznan.pl, @cezary\_mazurek

James McClure

Chris Mentzel @chrismentzel - firstname dot lastname at moore dot org

J. Yates Monteith, Clemson University

Pierre Neyron

Lucas Nussbaum (lucas.nussbaum@loria.fr)

Cody Permann

John Peterson

Quan Pham, University of Chicago

Marlon Pierce, marpierc@iu.edu, @marpierc

Heather Piwowar

Jason Riedy, Georgia Tech, http://fmrl.me/jasonriedy and many other places

Andreas Schreiber (@onyame)

William Scullin

Andrew Slaughter

Arfon Smith

James Spencer (@exp\_n)

Shel Swenson, University of Southern California (shel.swenson@usc.edu)

Fr\'{e}d\'{e}ric Suter

Andy Terrel (@aterrel, andy.terrel@gmail.com, NumFOCUS and Continuum Analytics)

George K. Thiruvathukal, Loyola University Chicago and Computing in Science and Engineering (IEEE CS/AIP) (gkt@cs.luc.edu)

John Towns

Erik Trainer

Matthew Turk (@powersoffour on Twitter, matthewturk@gmail.com)

Eric Van Wyk, University of Minnesota, evw@cs.umn.edu, @ericvanwyk

Colin C. Venters, c.venters@hud.ac.uk

Von Welch, @VonWelch

Nancy Wilkins-Diehr, wilkinsn@sdsc.edu

Theresa Windus (twindus@iastate.edu)

Rich Wolski

Lynn Zentner





\bibliographystyle{plain}

\bibliography{wssspe}
\end{document}

