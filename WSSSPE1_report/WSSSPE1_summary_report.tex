\documentclass[11pt, oneside]{amsart}

\usepackage{amsmath}
\usepackage{amssymb}

\usepackage[utf8]{inputenc}
\usepackage[T1]{fontenc}
\usepackage{lmodern}

\usepackage{color}
\usepackage{dcolumn}
\usepackage{float}
\usepackage{graphicx}
\usepackage{multirow}
\usepackage{rotating}
\usepackage{subfigure}
\usepackage{psfrag}
\usepackage{tabularx}
\usepackage[hyphens]{url}
\usepackage{wrapfig}
\usepackage{enumitem}
\usepackage{multicol}

%\setcounter{secnumdepth}{3}
%\setcounter{tocdepth}{3}


\usepackage[bookmarks, bookmarksopen, bookmarksnumbered]{hyperref}
\usepackage[all]{hypcap}
\urlstyle{rm}

\definecolor{orange}{rgb}{1.0,0.3,0.0}
\definecolor{violet}{rgb}{0.75,0,1}
\definecolor{darkgreen}{rgb}{0,0.6,0}
\definecolor{cyan}{rgb}{0.2,0.7,0.7}
\definecolor{blueish}{rgb}{0.2,0.2,0.8}

\newcommand{\todo}[1]{{\color{blue}$\blacksquare$~\textsf{[TODO: #1]}}}
\newcommand{\katznote}[1]{ {\textcolor{magenta}    { ***Dan:      #1 }}}
\newcommand{\gabnote}[1]{ {\textcolor{cyan}    { ***Gabrielle:     #1 }}}
\newcommand{\nchnote}[1]{  {\textcolor{orange}      { ***Neil: #1 }}}
\newcommand{\manishnote}[1]{  {\textcolor{violet}     { ***Manish: #1 }}}
\newcommand{\davidnote}[1]{  {\textcolor{darkgreen}      { ***David: #1 }}}
\newcommand{\note}[1]{ {\textcolor{red}    { #1 }}}
 
% Don't use tt font for urls
\urlstyle{rm}

% 15 characters / 2.5 cm => 100 characters / line
% Using 11 pt => 94 characters / line
\setlength{\paperwidth}{216 mm}
% 6 lines / 2.5 cm => 55 lines / page
% Using 11pt => 48 lines / pages
\setlength{\paperheight}{279 mm}
\usepackage[top=2.5cm, bottom=2.5cm, left=2.5cm, right=2.5cm]{geometry}
% You can use a baselinestretch of down to 0.9
\renewcommand{\baselinestretch}{0.96}

\sloppypar

\begin{document}

\title[]{Overview and Summary of the First Workshop on Sustainable Software for Science: Practice and Experiences (WSSSPE)}

\author{Authors TBD}

%\author{Daniel S. Katz$^{(1)}$, Gabrielle Allen$^{(2)}$, Neil Chue Hong$^{(3)}$, \\
%Manish Parashar$^{(4)}$, and David Proctor$^{(1)}$ }
%
%
%\thanks{{}$^{(1)}$ National Science Foundation, Arlington, VA, USA}
%
%\thanks{{}$^{(2)}$ Skolkovo Institute of Science and Technology, Moscow, Russian Federation}
%
%\thanks{{}$^{(3)}$ Software Sustainability Institute, University of Edinburgh, Edinburgh, UK}
%
%\thanks{{}$^{(4)}$ Rutgers Discovery Informatics Institute, Rutgers University, New Brunswick, NJ, USA}


\begin{abstract}
This technical report discusses the First Workshop on Sustainable Software for Science: Practice and Experiences (WSSSPE). ...
\end{abstract}


\maketitle

\note{papers and presentations and URLs should be references - see
  figshare-web in .bib file for URL style example}

People who have volunteered to work on this report: (with * for actual
contributors - \note{if you contribute, add a * by your name})

The workshop organizers:
\begin{itemize}
\item * Daniel S. Katz $\langle$\url{dkatz@nsf.gov}$\rangle$
\item Gabrielle Allen $\langle$\url{gdallen@illinois.edu}$\rangle$
\item Neil Chue Hong $\langle$\url{N.ChueHong@software.ac.uk}$\rangle$
\item Manish Parashar $\langle$\url{parashar@rutgers.edu}$\rangle$
\item David Proctor $\langle$\url{dproctor@nsf.gov}$\rangle$
\end{itemize}


Additional volunteers:
\begin{itemize}
\item Chris Mattmann $\langle$\url{chris.a.mattmann@jpl.nasa.gov}$\rangle$
\item * Ketan Maheshwari $\langle$\url{ketancmaheshwari@gmail.com}$\rangle$
\item Marlon Pierce $\langle$\url{marpierc@iu.edu}$\rangle$
\item Colin Venters $\langle$\url{C.Venters@hud.ac.uk}$\rangle$
\item Suresh Marru $\langle$\url{smarru@iu.edu}$\rangle$
\item Lynn Zentner $\langle$\url{lzentner@purdue.edu}$\rangle$
\item Anne Elster $\langle$\url{anne.elster@gmail.com}$\rangle$
\item * Shel Swenson $\langle$\url{mdswenso@usc.edu}$\rangle$
\item Andy Ray Terrel $\langle$\url{andy.terrel@gmail.com}$\rangle$
\item Abani Patra $\langle$\url{abani.patra@gmail.com}$\rangle$
\item * Nancy Wilkins-Diehr $\langle$\url{wilkinsn@sdsc.edu}$\rangle$
\item James Spencer $\langle$\url{j.spencer@imperial.ac.uk}$\rangle$
\item * Frank L\"{o}ffler $\langle$\url{knarf@cct.lsu.edu}$\rangle$
\item * James Hetherington $\langle$\url{j.hetherington@ucl.ac.uk}$\rangle$
\item * Sou-Cheng (Terrya) Choi $\langle$\url{sou.cheng.terrya.choi@gmail.com}$\rangle$
\item James Howison $\langle$\url{james@howison.name}$\rangle$
\item * Bruce Berriman $\langle$\url{gbb@ipac.caltech.edu}$\rangle$
\item Hilmar Lapp $\langle$\url{hlapp@nescent.org}$\rangle$
\item * Marcus D. Hanwell $\langle$\url{marcus.hanwell@kitware.com}$\rangle$
\item Lucas Nussbaum $\langle$\url{lucas.nussbaum@loria.fr}$\rangle$
\item Matthew Turk $\langle$\url{matthewturk@gmail.com}$\rangle$
\end{itemize}

The original document is
\url{https://docs.google.com/document/d/1eVfioGNlihXG_1Y8BgdCI6tXZKrybZgz5XuQHjT1oKU/edit?pli=1#}
(but can no longer be edited).  Note that the original document has
comments in addition to text.


\pagebreak

\section*{Executive Summary}

\todo{1 page summary goes here}

\pagebreak

\section{Introduction}

The First Workshop on Sustainable Software for Science: Practice and
Experiences (WSSSPE1,
\url{http://wssspe.researchcomputing.org.uk/WSSSPE1}) was held on
Sunday, 17 November 2013, in conjunction with the 2013 International
Conference for High Performance Computing, Networking, Storage and
Analysis (SC13, \url{http://sc13.supercomputing.org}).

Because progress in scientific research is dependent on the quality and
accessibility of software at all levels, it is now critical to address many
challenges related to the development, deployment, and maintenance of reusable
software.
%\note{Ketan: What are the levels in scientific process where software
%is required. It might be interesting to outline these stages and types of
%software required in each stage.}
In addition, it is essential that scientists,
researchers, and students are able to learn and adopt software-related skills
and methodologies. Established researchers are already acquiring some of these
skills, and in particular a specialized class of software developers is
emerging in academic environments who are an integral and embedded part of
successful research teams. The WSSSPE workshop was intended to provide a forum
for discussion of the challenges, including both positions and experiences. The
short papers and discussion have been archived to provide a basis for continued
discussion, and the workshop has fed into the collaborative writing of this
journal publication. An estimate of 90 to 150 participants were present 
throughout the day. Additional papers based on extended versions of workshop
submissions are expected.  The level of interest in the workshop has led the
organizers, working with some of the submitters and attendees, to plan two
additional workshops (at the 2014 SciPy and SC14 conferences) and to turn the
workshop website into a community website that can be used as a focus for
further discussion and progress.  Additionally, a minisymposium at 2014 SIAM
Annual Meeting on ``Reliable Computational
Science'' that aims to further explore some of the key issues raised in WSSSPE
is co-organized by a WSSSPE1 participant.

Before the WSSSPE1 workshop took place, the organizers self-published
a paper~\cite{WSSSPE1-pre-report} to document the process of
organizing and advertising the workshop, collecting and reviewing the
papers, and putting together the agenda.  Section~\ref{sec:process} is
based on that paper. The remainder of this paper is based on the
workshop itself, as documented in a set of collaborative notes taken
during the workshop~\cite{WSSSPE1-google-notes}, including discussion
about the keynote presentations (\S\ref{sec:keynotes}), the three
panels (\S\ref{sec:devel}-\ref{sec:community}), other discussion
(\S\ref{sec:other}), cross-cutting issues (\S\ref{sec:cross-cutting}),
use cases (\S\ref{sec:use-cases}), and conclusions
(\S\ref{sec:conclusions}).
 
\section{Workshop Process and Agenda} \label{sec:process}

WSSSPE1 was organized in collaboration of the relatively small group
of five organizers and a larger peer-review committee with 36
members. This committee had early influence on the organization, e.g.,
on the specific call for papers.

The workshop call for papers included:

\begin{quote}
In practice, scientific software activities are part of an ecosystem
where key roles are held by developers, users, and funders. All three
groups supply resources to the ecosystem, as well as requirements that
bound it. Roughly following the example of NSF's Vision and Strategy
for Software
(\url{http://www.nsf.gov/publications/pub_summ.jsp?ods_key=nsf12113})~\cite{NSF_software_vision},
the ecosystem may be viewed as having challenges related to:

\begin{itemize}[leftmargin=0.2in]
\item the development process that leads to new (versions of) software
\begin{itemize}[leftmargin=0.2in]
\item how fundamental research in computer science or
  science/engineering domains is turned into reusable software
\item software created as a by-product of research
\item impact of computer science research on the development of
    scientific software and vice versa
\end{itemize}
\item the support and maintenance of existing software, including
  software engineering
\begin{itemize}[leftmargin=0.2in]
\item governance, business, and sustainability models
\item the role of community software repositories, their operation and
  sustainability
\end{itemize}
\item the role of open source communities or industry
\item use of the software
\begin{itemize}[leftmargin=0.2in]
\item growing communities
\item reproducibility, transparency needs that may be unique to science
\end{itemize}
\item policy issues, such as
\begin{itemize}[leftmargin=0.2in]
\item measuring usage and impact
\item software credit, attribution, incentive, and reward
\item career paths for developers and institutional roles
\item issues related to multiple organizations and multiple countries,
  such as intellectual property, licensing, etc.
\item mechanisms and venues for publishing software, and the role of
  publishers
\end{itemize}
\item education and training
\end{itemize}

\end{quote}

Based on the goal of encouraging a wide range of submissions from
those involved in software practice, ranging from initial thoughts and
partial studies to mature deployments, the organizers wanted to make
submission as easy as possible. The call for papers stated:

\begin{quote}
We invite short (4-page) position/experience reports that will be used
to organize panel and discussion sessions. These papers will be
archived by a third-party service, and provided DOIs. We encourage
submitters to license their papers under a Creative Commons license
that encourages sharing and remixing, as we will combine ideas (with
attribution) into the outcomes of the workshop.  An interactive site
will be created to link these papers and the workshop discussion, with
options for later comments and contributions. Contributions will be
peer-reviewed for relevance and originality before the links are added
to the workshop site; contributions will also be used to determine
discussion topics and panelists. We will also plan one or more papers
to be collaboratively developed by the contributors, based on the
panels and discussions.
\end{quote}

58 submissions were received, and almost all submitters used either
arXiv~\cite{arXiv-web} or figshare~\cite{figshare-web} to self-publish
their papers.

A peer review process followed the submissions, where
the 58 papers received 181 reviews, an average of 3.12 reviews per
paper. Reviews were completed using a Google form, which allowed
reviewers to choose papers they wanted to review, then to provide
general comments and scores on relevance to the organizers and
to the authors. The review reports enabled the organizers to decide
which papers to associate with the workshop, and allowed the authors
to improve their papers.

The organizers decided to list 54 of the papers as significant
contributions to the workshop, a very high acceptance rate, but one
that is reasonable, given the goal of broad participation and the fact
that the reports were already self-published. The papers were
grouped into three main categories, namely \emph{Developing and
Supporting Software}, \emph{Policy}, and \emph{Communities}. Each
subject area was associated with a panel and discussion at the
workshop.

The workshop itself consisted of two keynote presentations and the
three panels/discussion sessions. The panels were organized based on a
classification of the workshop submissions into three categories,
following the themes of the call for papers as modified by the areas
of the submissions. Three to four representatives from each submission
category were appointed as panelists, and assigned to read a subset of
the paper in that category and then discuss them in the panel.


\section{Keynotes \note{lead: Sou-Cheng (Terrya) Choi} } \label{sec:keynotes}


The WSSSPE1 workshop began with two keynote presentations.


\subsection{A Recipe for Sustainable Software, Philip E. Bourne.} \label{sec:keynote1}

The first keynote~\cite{WSSSPE1-keynote1} was delivered by Philip
E. Bourne of University of California, San Diego.  Bourne is a basic
biomedical scientist but has formed four software companies. He co-founded 
PLOS Computational Biology~\cite{plos-web} and helped develop the RCSB PDB~\cite{pdb-web}.  
He is working on automating three-dimensional visualizations of cell
contents and molecular structures, a problem that has not been solved
and when done, would serve as a key function of software in biomedical
sciences.

Bourne's keynote presentation was entitled ``A Recipe for Sustainable
Software,'' developed based on his experiences.  He emphasized that
sustainability for software ``does not just mean more money from
Government'' (see also Section~\ref{sec:cross-cutting}).  Other
factors to consider, he mentioned, include costs of production, ease
of maintainence, community involvement, and distribution channels.

In places, Bourne said, development in science has improved thanks to
open source and GitHub~\cite{github-web}, but for the most part
remains arcane. He argued that we can learn much from the App Store
model about interfaces, ratings, and so on. He also mentioned
BioJava~\cite{biojava-web} and Open Science Data Cloud~\cite{osdc-web}
as distribution channels.  On a related note, Bourne observed a common
evolutionary pathway for computational biology projects, from data
archive to analytics platform to educational use, and suggested that
use of scientific software for outreach might be the final step.

Bourne shared with the audience a few real challenges he
encountered. First, also an anecdote, he has looked into
reproducibility in computational biology, but has concluded that ``I
have proved I cannot reproduce research from my own
lab.''~\cite{Veretnik}

Another problem he experienced was staff retention with respect to
private organizations which reward those combining research and
software expertise (the ``Google Bus''). However, he is a strong fan
of software sustainability through public-private partnerships. He
noted that making a successful business from scientific software alone
is very rare: founders overvalue while customers undervalue. He noted
that to last, an open source project needs a minimal funding
requirement even with a vibrant community --- goodwill only goes so
far if one is being paid to do something else.  He talked about grant
schemes of relevance in the U.S., particularly with regard to
technology transfer~\cite{sbir-web, fased-web}.

He also had problems with selling research software: the technology
transfer office in his university wanted huge intellectural property
reach through, whereby they would get a share of profits from drugs
developed by pharmaceutical companies who use the software.  He was
aware this was unrealistic but the technology transfer office insisted
for a while. He wants to push a one-click approach for customers to
purchase university-written software.

He then presented the usual arguments on directly valuing software as
a research output alongside papers. Readers will be familiar with this
debate.  An interesting reference provided by him
is~\cite{peer-review-code}, which explores involving software
engineers in the review process of scientific code.

On the notion of \emph{digital enterprise}, where information
technology (IT) underpins the whole of organizational activities, he
contended that universities are way behind the curve. In
particular, he highlighted the separation of research, teaching, and
administration into silos without a common IT framework as a blocker
to many useful organizational innovations: ``University 2.0 is yet to
happen.'' He spoke of a circumstance where someone had used an
algorithm developed for computational biology in marketing.
The role of an institution is important in this space. He argued that
funders can help train institutions, not just individuals, in this regard.

He concluded with a reference to his paper~\cite{bourne_ten} and
argued that computational scientists ``have a responsibility to
convince their institutions, reviewers, and communities that software is
scholarship, frequently
more valuable than a research article,'' a point with which we strongly agree.

\subsection{Scientific Software and the Open Collaborative Web, Arfon Smith} \label{sec:keynote2}

The second keynote~\cite{WSSSPE1-keynote2} was delivered by Arfon
Smith of GitHub.

Smith's talk started with an example from data reduction in Astronomy,
where he needed to remove interfering effects from the device. He
built a ``bad pixel mask,'' and realized that while it was persistent,
there was no way or practice of sharing this with the
data. Consequently many researchers repeated the same calculations. He
estimated that 13 person-years were wasted in this redundant
calculation.

``Why didn't we do better?''  Smith asked of this practice. He argued
this was because we were taught to focus on immediate research
outcomes and not on continuously improving and building on tools for
research. He then asked, when we do know better, why we do not act any
different. He argued that it was due to incentives and their lack:
only the immediate products of research, not the software, are valued.
He referenced Victoria Stodden's talk at
OKCon~\cite{okcon-stodden-talk} which he said argued these points
well.

C. Titus Brown~\cite{ged-web}, a WSSSPE1 contributor and participant,
argued that with regard to reusable software, ``we should just start
doing it.'' In this regard Smith replied that documentation should be
``treated as a first class entity.''  He noted that the open source
community has excellent cultures of code reuse, for example,
RubyGems~\cite{rubygems-web}, PyPI~\cite{pypi-web}, and
CPAN~\cite{cpan-web}, where there is effectively low-friction
collaboration through the use of repositories. This has not happened
in highly numerical, compiled language scientific software.  An
exception he cited as a good example of scientific projects using
GitHub is the EMCEE Markov Chain Monte Carlo project~\cite{emcee-web}
by Dan Foreman-Mackey and contributors.

He argued that GitHub's \emph{Pull Request} code review mechanism
facilitates such collaboration, by allowing one to code first, and
seek review and merge back into the trunk later.

``Open source is \ldots reproducible by necessity,'' Smith quoted
Fernando Perez~\cite{perez-open-src-reproducible}, explaining that
reproducibility is a prerequisite for remote collaboration.  He
pointed out that GitHub could propel the next stage of web
development, i.e., ``the collaborative web,'' following on from the
social web of Facebook.

In conclusion Smith reiterated the importance of establishing
incentive models for open contributions and tool builders, for
example, meaningful metrics and research grants such as
\cite{NSF_software_vision}. He urged computational scientists to
collabroate and share often their research reports, teaching
materials, code, as well as data by attaching proper licenses.

\section{Developing and Supporting Software \note{lead: Marcus Hanwell, contributors: Suresh Marru}} \label{sec:devel}

\todo{section should include what was presented and what was discussed}

The panel on developing and supporting software examined the challenges
around scientific software development/support, mainly focused on research
groups that also produce code in various forms. There was widespread agreement
that developing and maintaining software is hard, but best practices can help.
Several stated that documentation is not just for users, and paying attention
to API documentation, tutorials for building and deploying software, along with
documented development practices can be very helpful in bringing  new developers
into a project efficiently.

There was discussion that backward compatibility is not always desirable, and it
can be very costly. This must be balanced with the aims of a given project, and how
many other projects depend on and use the code when backwards incompatible
changes are to be made. There are many examples in the wider open source
software world of strategies for dealing with this, and again best practices
can go a long way to mitigating issues around backwards compatibility. Many
projects live with sub-optimal code for a while, and may allow backwards
compatibility to be broken at agreed upon development points, such as a major
release for a software library.

Communities are extremely important in software projects, and both their
building and continued engagement need attention during the project life cycle.
Several of the submitted papers discussed how communities have been built around
projects, and what is needed to enable a project to grow. Among these are public
source code hosting, mailing lists, documentation, wikis, bug trackers, software
downloads, continuous integration, software quality dashboards, and of course,
a general web presence to tie all of these things together. Questions were posed
around users not answering the questions of other users. Several participants
offered counterexamples, such as lists where developers do not participate in
mailing lists as much due to users being more active, or whether the ``core
team'' can end up setting unrealistic expectations by doing too much. Team Geek\todo{add cite}
and Turk's paper on scaling code in the human dimension\todo{add cite}  discuss how development
lists that are welcoming to people actually have many more people contributing.

Recruiting and/or retaining personnel in this area is hard,
with one of the major reasons being no long-term career paths (especially when
compared with industry). How should software development be balanced with
research? It is apparent that things are beginning to improve, such as
incentives for software development in the form of altmetrics, tenure committee
consideration, and NSF ``products'' vs ``publications''. It was noted it can be very
difficult to measure where people are using small bits of your code.

There were 13 articles about different experiences in this area, but little
about GUI testing, performance, scalability, or agile development practices.
There were several unique perspectives about issues such as managing API
changes, using the same best practices for software as data, and going beyond
simply ``slapping an OSI-approved license on code.''

The question of what ``sustainable'' means in the context of software was
raised (see \S\ref{sec:cross-cutting}.) The resources at \url{http://oss-watch.ac.uk/resources/ssmm} discuss what
to look for when choosing software for procurement, or reuse in further software
development. Regardless of the license and development model that will be used,
the future of the project must be considered. Even if a particular piece of
software satisfies today's requirements, will it continue to do so in the
future? Important questions include whether the supplier will still be around in five
years time, will it still care for its customers needs, will it be responsive to
bug reports and feature requests? Should you tie your investment to a single
supplier using a proprietary product, or ensure the project uses an OSI-approved
license, and can outlive any single entity if the software is still important.

What is the overarching goal of sustainability? Is it reproducible science,
persistence, quality, something else? How should success be measured in this
context? Is there some metric that can be used to determine when you have
achieved sustainable software, or is this an ongoing process with no clear
endpoint. For truly sustainable software there should be no endpoint, as the
software products continue to be used and useful beyond the single institution,
grant, and developer/development team. Sustainability must be addressed
throughout the project life cycle.

What about actual software engineering principles, such as modularity and
extensibility? This is how industry maintains software, and ensures it continues
to be useful. Everyone \note{everyone in the audience?  paper authors?  software developers in general?} thinks that you can't rewrite your software, but if it is
modular then you can keep systems up to date. Extensibility will keep you
relevant if it is built into the project. One counterpoint raised by Jason Riedy
was that trying to take advantage of the latest and greatest hardware often
makes this painful, hence the lack of papers mentioning ``GPUs and exotic
hardware''.

The question of whether funders, such as the NSF, can mandate software plans in
much the same way as they do data management plans. Daniel Katz responded that
software is supposed to be described as part of the NSF data management plan,
and that in their definition, data includes software. A comment from Twitter
(@biomickwatson) raised the issue that this requires reviewers and funders who
understand the answers that are given in these plans. Daniel Katz responded
that in programs focused on software or data this can be done effectively, but
agreed that in more general programs this is indeed a problem.

The papers submitted to this panel, and several others, include lots of general
recommendations for processes, practices, tools, etc. One of the papers
suggested that a ``Software Sustainability Institute'' should be vested with
the authority to develop standardized quality processes, a central common
repository, central resources for services and consulting, a think tank of
sorts, and a software orphanage center. The idea received some resistance,
stating difficulties in acting as a central common repository with so many
compelling alternatives, and as an orphanage. The point for centralization of
communication/point of contact might be reasonable, with the statement that
``vested with authority'' is perhaps too strong, but ``providing tools if
needed'' might be more appropriate. The wider open source community is moving
to a model of distributed version control (Git, Mercurial, Bazaar, etc), rather
than more centralized, as discussed in the second keynote.

Some of the questions raised after the panel discussion are:

Rather than teaching developers about domain science, or domain scientists about
software development practices, why don't we teach both communities to
collaborate more effectively? Can this be done without teaching each side a
little of the other to enable communication, with a response that it really is
not two binary communities, but a spectrum with useful roles in the middle.
Someone else asked if a developer can be effective without being part of the
domain community, with responses that this really depends on the specific
problem---translators and T-shaped people are important. Why aren't academic
communities taught how to evaluate cross-disciplinary work well?

Is there a role for the growing science of the team science field in this? There
is overlap between the communities, with support for virtual organizations,
tool development, etc. How can we make time in an already crowded schedule to
introduce these topics to students? Should they be introduced through lab
classes as in ``real sciences''?

Are there significant differences in projects that have been running for 1, 3,
5, or 10+ years? Are there shared experiences for projects of a similar stage
of maturity? It was noted that computing and communication have changed
significantly over the past decade, and many of the experiences are tied to
the history of computing and communication. See the history of GCC, Emacs, or
the Visualization Toolkit for examples. Others felt that computing has changed
less, but communication and the widespread availability of tools has. It was
noted that email lists, websites, chat rooms, version control, virtual and
physical meetings are all over 20 years old.

There was debate that while some of the basics of computing may be fairly
similar, that the tools commonly used for computing have changed quite
significantly. Reference was made to Perl, which was commonly used, giving way
to whole new languages, such as Python, for gluing things together and how this
induces many students into entirely rewriting the scaffolding, leaving the old
to rot and the experiments to become non-reproducible as the tools change.
Jason Riedy stated that he had been guilty of this in the past. There was
discussion of this tendency along with the enormous differences in the speed and
ease of sharing---having to ship tapes around (GCC, Emacs, etc) as opposed to
the immediate sharing of the latest development in CVS, Subversion, Git,
Mercurial, Bazaar, etc.

The question was also posed as to whether the distinction between researcher
and developer is sensible, with James Hetherington commenting that in the UK
they have been examining a more nuanced view of research software engineers
and researcher developers. Should this be less of a contract relationship, and
more of a collaborative relationship. This is also at the core of the business
model that Kitware presented in its submission to the workshop. Are other
ingredients missing such as applied mathematicians? Should this be defined more
in terms of skill sets rather than roles and/or identities? This builds on the
comments from Vaidy Sunderam that scientists are generally good writers, and
have mathematical skills, so why can't they learn software engineering
principles?

Miller commented that all of the infrastructure that sits around a new
algorithm that we need to make it useful and sustainable requires different
skill sets than the algorithm developer. Friere commented that there are no
good career paths for people with broad skills, no incentives for them to
continue in these roles. There was debate around people doing what interests
them, and learning computing leaves people cold, but is it that it leaves the
people who find career paths in academia cold versus the full spectrum of
people involved in research? Is this also caused by poor teaching, or that the
perceived benefits for doing this are too small? It could also be attributed to
their focus being on science, not software engineering, or do the people with
the passion for software engineering in science simply have no viable career
path and either adapt or seek out alternate career paths.

\subsection*{Papers}

\subsubsection*{Development Experiences}

\begin{itemize}

\item Mark C. Miller, Lori Diachin, Satish Balay, Lois Curfman
  McInnes, Barry Smith. Package Management Practices Essential for
  Interoperability: Lessons Learned and Strategies Developed for
  FASTMath \cite{Miller_WSSSPE}

\item Karl W. Broman, Thirteen years of R/qtl: Just barely sustainable
  \cite{Broman_WSSSPE}

\item Charles R. Ferenbaugh, Experiments in Sustainable Software
  Practices for Future Architectures \cite{Ferenbaugh_WSSSPE}

\item Eric G Stephan, Todd O Elsethagen, Kerstin Kleese van Dam, Laura
  Riihimaki. What Comes First, the OWL or the Bean?
  \cite{Stephan_WSSSPE}

\item Derek R. Gaston, John Peterson, Cody J. Permann, David Andrs,
  Andrew E. Slaughter, Jason M. Miller, Continuous Integration for
  Concurrent Computational Framework and Application Development
  \cite{Gaston_WSSSPE}

\item Anshu Dubey, B. Van Straalen. Experiences from Software
  Engineering of Large Scale AMR Multiphysics Code Frameworks
  \cite{Dubey_WSSSPE}

\item Markus Blatt. DUNE as an Example of Sustainable Open Source
  Scientific Software Development \cite{Blatt_WSSSPE}

\item David Koop, Juliana Freiere, Cl\'{a}udio T. Silva, Enabling
  Reproducible Science with VisTrails~\cite{Koop_WSSSPE}

\item Sean Ahern, Eric Brugger, Brad Whitlock, Jeremy S. Meredith,
  Kathleen Biagas, Mark C. Miller, Hank Childs, VisIt: Experiences
  with Sustainable Software \cite{Ahern_WSSSPE}

\item Sou-Cheng (Terrya) Choi. MINRES-QLP Pack and Reliable
  Reproducible Research via Staunch Scientific Software
  \cite{Choi_WSSSPE}

\item Michael Crusoe, C. Titus Brown. Walking the talk: adopting and
  adapting sustainable scientific software development processes in a
  small biology lab \cite{Crusoe_WSSSPE}

\item Dhabaleswar K. Panda, Karen Tomko, Karl Schulz, Amitava Majumdar.
The MVAPICH Project: Evolution and Sustainability of an Open Source
Production Quality MPI Library for HPC \cite{Panda_WSSSPE}

\item Eric M. Heien, Todd L. Miller, Becky Gietzel, Louise
  H. Kellogg. Experiences with Automated Build and Test for
  Geodynamics Simulation Codes \cite{Heien_WSSSPE}

\end{itemize}

\subsubsection*{Deployment, Support, and Maintenance of Existing Software}

\begin{itemize}

\item Henri Casanova, Arnaud Giersch, Arnaud Legrand, Martin Quinson,
  Fr\'{e}d\'{e}ric Suter. SimGrid: a Sustained Effort for the
  Versatile Simulation of Large Scale Distributed
  Systems~\cite{Casanova_WSSSPE}

\item Erik Trainer, Chalalai Chaihirunkarn, James Herbsleb. The Big
  Effects of Short-term Efforts: A Catalyst for Community Engagement
  in Scientific Software \cite{Trainer_WSSSPE}

\item Jeremy Cohen, Chris Cantwell, Neil Chue Hong, David Moxey,
  Malcolm Illingworth, Andrew Turner, John Darlington, Spencer
  Sherwin. Simplifying the Development, Use and Sustainability of HPC
  Software \cite{Cohen_WSSSPE}

\item Jaroslaw Slawinski, Vaidy Sunderam. Towards Semi-Automatic
  Deployment of Scientific and Engineering Applications
  \cite{Slawinski_WSSSPE}

\end{itemize}

\subsubsection*{Best Practices, Challenges, and Recommendations}

\begin{itemize}

\item Andreas Prli\'{c}, James B. Procter. Ten Simple Rules for the
  Open Development of Scientific Software \cite{Prlic_WSSSPE}

\item Anshu Dubey, S. Brandt, R. Brower, M. Giles, P. Hovland,
  D. Q. Lamb, F. Löffler, B. Norris, B. O'Shea, C. Rebbi, M. Snir,
  R. Thakur, Software Abstractions and Methodologies for HPC
  Simulation Codes on Future Architectures \cite{Dubey2_WSSSPE}

\item Jeffrey Carver, George K. Thiruvathukal. Software Engineering
  Need Not Be Difficult \cite{Carver_WSSSPE}

\item Craig A. Stewart, Julie Wernert, Eric A. Wernert, William
  K. Barnett, Von Welch. Initial Findings from a Study of Best
  Practices and Models for Cyberinfrastructure Software Sustainability
  \cite{Stewart_WSSSPE}

\item Jed Brown, Matthew Knepley, Barry Smith. Run-time extensibility:
  anything less is unsustainable \cite{Brown_WSSSPE}

\item Shel Swenson, Yogesh Simmhan, Viktor Prasanna, Manish Parashar,
  Jason Riedy, David Bader, Richard Vuduc. Sustainable Software
  Development for Next-Gen Sequencing (NGS) Bioinformatics on Emerging
  Platforms \cite{Swenson_WSSSPE}

\end{itemize}

\subsection{Research or Reuse?}

Discussion of software produced as a by-product versus software
developed for reuse. How does this change the project, who develops
the code, growth beyond 1--2 person projects to larger projects with
diverse set of consumers and contributors. Modular and extensible code
versus working for current research problem---tackled by same people,
or a collaborative team?

\subsection{Defining Sustainability for Scientific Software}

What is sustainable? What are the best practices, can workshops fund
meetings to help define and improve best practices in these areas?
Software plans from the funding agencies? This topic is expanded upon
in \S\ref{sec:cross-cutting}, since it was discussed up in multiple
parts of the WSSSPE1 workshop.

\subsection{Training Scientists to Develop and Support Software}

Discussion of the evolving role of scientists, and/or others that fill
these roles.

\subsection{Software Process, Code Review, Automation, Reproducibility}

There are a lot of tools out there, but few are currently used. Look
at what some projects have found successful, and how to automate as
much as possible to reduce additional overhead.

\subsection{Software and Data Licensing}

Its impact on how and where research products are used, who can
collaborate and what patterns have worked/not worked in existing
communities.

\subsection{Funding, Sustainability Beyond the First Grant/Institution}

How initial software development is funded, moving to follow up
projects, maintenance, community growth, collaborating with other
institutions, labs, industry, internationally. Contract relationship
versus collaborative development between scientists and software
developers.

\subsection{Training Others to use Software}

Who creates training materials, how are they distributed, integration
into courses when projects see very wide application in research.

\section{Policy \note{lead: Colin Venters}} \label{sec:policy}

\todo{section should include what was presented and what was discussed}

\subsection*{Papers}

\subsubsection*{Modeling Sustainability}

\begin{itemize}

\item Coral Calero, M. Angeles Moraga, Manuel F. Bertoa. Towards a
  Software Product Sustainability Model \cite{Calero_WSSSPE}

\item Colin C. Venters, Lydia Lau, Michael K. Griffiths, Violeta
  Holmes, Rupert R. Ward, Jie Xu. The Blind Men and the Elephant:
  Towards a Software Sustainability Architectural Evaluation Framework
  \cite{Venters_WSSSPE}

\item Marlon Pierce, Suresh Marru, Chris Mattmann. Sustainable
  Cyberinfrastructure Software Through Open Governance
  \cite{Pierce_WSSSPE}

\item Daniel S. Katz, David Proctor. A Framework for Discussing
  e-Research Infrastructure Sustainability \cite{Katz_WSSSPE}

\item Christopher Lenhardt, Stanley Ahalt, Brian Blanton, Laura
  Christopherson, Ray Idaszak. Data Management Lifecycle and Software
  Lifecycle Management in the Context of Conducting Science
  \cite{Lenhardt_WSSSPE}

\item Nicholas Weber, Andrea Thomer, Michael Twidale. Niche Modeling:
  Ecological Metaphors for Sustainable Software in Science
  \cite{Weber_WSSSPE}

\end{itemize}

\subsubsection*{Credit, Citation, Impact}

\begin{itemize}

\item Matthew Knepley, Jed Brown, Lois Curfman McInnes, Barry
  Smith. Accurately Citing Software and Algorithms Used in
  Publications \cite{Knepley_WSSSPE}

\item Jason Priem, Heather Piwowar. Toward a comprehensive impact
  report for every software project \cite{Priem_WSSSPE}

\item Daniel S. Katz. Citation and Attribution of Digital Products:
  Social and Technological Concerns \cite{Katz2_WSSSPE}

\item Neil Chue Hong, Brian Hole, Samuel Moore. Software Papers:
  improving the reusability and sustainability of scientific software
  \cite{Chue_Hong_WSSSPE}

\end{itemize}

In addition, the following paper from another area were also discussed
in this area.

\begin{itemize}

\item Frank L\"{o}ffler, Steven R. Brandt, Gabrielle Allen and Erik
  Schnetter. Cactus: Issues for Sustainable Simulation Software
  \cite{Loffler_WSSSPE}

\end{itemize}

\subsubsection*{Reproducibility}

\begin{itemize}

\item Victoria Stodden, Sheila Miguez. Best Practices for
  Computational Science: Software Infrastructure and Environments for
  Reproducible and Extensible Research \cite{Stodden_WSSSPE}

\end{itemize}

\subsubsection*{Implementing Policy}

\begin{itemize}

\item Randy Heiland, Betsy Thomas, Von Welch, Craig Jackson. Toward a
  Research Software Security Maturity Model \cite{Heiland_WSSSPE}

\item Brian Blanton, Chris Lenhardt, A User Perspective on Sustainable
  Scientific Software \cite{Blanton_WSSSPE}

\item Daisie Huang, Hilmar Lapp. Software Engineering as
  Instrumentation for the Long Tail of Scientific Software
  \cite{Huang_WSSSPE}

\item Rich Wolski, Chandra Krintz, Hiranya Jayathilaka, Stratos
  Dimopoulos, Alexander Pucher. Developing Systems for API Governance
  \cite{Wolski_WSSSPE}

\end{itemize}


\subsection{Career Tracks for Scientific Software Developers}

How to ensure software is sustainable by ensuring there are career
paths for developers.

What are the possible career paths for a specialist software developer working as part of a scientific research group?

\section{Communities \note{lead: Matthew Turk, contributors: Nancy Wilkins-Diehr, Chris Mattmann, Suresh Marru, Frank L\"{o}ffler, Andy Terrel}} \label{sec:community}


\todo{section should include what was presented and what was discussed}

There were a number of papers categorized under the Communities banner and so two presenters each summarized half of the submissions in this area.

\subsection{Communities Part 1}

This collection of papers was summarized by Karen Cranston.

One paper by Maheshwari et al.~\cite{Maheshwari_WSSSPE} that was
discussed was about the role of technological catalysts in overall e-research
process.  The paper focuses on ``technology catalysts" and their role in modern
scientific research process. Technology catalyst could be defined as a role
played by an individual with a knowhow of technological advancements tasked
with user engagement with a goal of enacting scientific or engineering
applications using suitable tools and techniques to take advantage of
technological capabilities and benefitting applications.

One of the tasks of catalysts is to seek community collaborations for new
applications and user engagement thus benefitting both: science, by effective
running of scientific codes over computational infrastructure and technology,
by conducting research and seeking findings to improve technology.

The particular engagements described in the paper came up from authors work as
postdoctoral researcher at Cornell and Argonne. Interaction with the scientific
communities in both institutions resulted in these collaborations.

\subsection{Communities Part 2}

Nancy Wilkins-Diehr summarized 6 papers in ``Communities Part 2.'' The Christopherson paper outlines the degree to which research
relies on high quality software. There are often barriers and a lack of suitable incentives for researchers to embrace software engineering principals.
The Water Science Software Institute is working to lower some of these barriers through an Open Community Engagement Process.
This is a 4-step iterative development process that incorporates Agile development principals.

\begin{itemize}

\item Step 1 - Design - thorough discussion of research questions
\item Step 2 - Develop working code
\item Step 3 - Refine based on new requirements
\item Step 4 - Publish open source

\end{itemize}

The Christopherson paper reports on the application of Steps 1-3 to a computational modeling framework developed in the 1990s.
Step 1 was realized as a 2-day, in person specifications meeting and code walk-through. Step 2 was a 5-day hackathon to develop working code.
Step 3 was a 3-day hackathon to refine the code based on new requirements. The team worked on small, low-risk unit of
code. It was challenging, revealed unanticipated obstacles, programmers had to work together, and experimentation was encouraged.

The paper summarized recommendations to others wishing to engage in this or a similar process. Start small and gradually
build toward more complex objectives. This is consistent with Agile development. Refactor before adding new functionality.
Approach development as a learning experience. Welcome experimentation, and treat mistakes as a natural part of the learning process.
Repeat Step 1 activities before all hackathons to develop consensus before coding. This allows coding to be the focus of subsequent hackathons.
In higher risk situations, provide additional time for Step 1 activities. The Christopherson team recommends a minimum of two months.
Ensure any newcomers receive some form of orientation prior to the hackathon, such as a code walkthrough or system documentation.
Co-locate rather than collaborating remotely whenever feasible.

The Pierce paper described how science gateways can provide a user-friendly entry to complex cyberinfrastructure. The paper opens with a description of the explosion of use of just a few gateways.
Over 3.5 years, more than 7,000 biologists have run phylogenetic codes on supercomputers using the CIPRES Science Gateway. In 1 year, over 120 scientists from 50 institutions used the UltraScan Science Gateway,
increasing the sophistication of analytical ultracentrifugation experiments worldwide. The new Neuroscience Gateway (NSG) registered 100+
users who used 250,000 CPU hours in only a few months.

Gateways, however, need to keep operational costs low and can often make use of common components. Science Gateway Platform as a Service (Sci-GaP)
delivers middleware as a hosted, scalable third-party service while domain developers focus on user interfaces and domain-specific data in the gateway.
The middleware handles things like authentication, application installation and reliable execution and help desk support.

One key to SciGaP is its openness. While based on Apache Airavata project and the CIPRES Workbench Framework, community contributions are encouraged because of its open source, open governance and open operations policies.
The goal is robust, sustainable infrastructure with a cycle of development that improves reliability and prioritizes
stakeholder requirements. The project is leveraging Internet2's Net+ mechanisms for converting SciGaP and its gateways into commodity services.

The Zentner paper describes experiences and challenges with the large nanoHUB.org community. They define community as
a ``body of persons of common and especially professional interests scattered through a larger society.'' This makes support
challenging because of the diversity of viewpoints and needs. The group constantly examines its policies to determine
whether they are indirectly alienating part of the community or encouraging particular types of use.

nanoHUB's 10-year history with over 260,000 users annually provides a lot of data to analyze. 4000 resources contributed by
1000 authors. nanoHUB serves both the research and education community and the contribution model allows researchers to get
their codes out into the community and in use in education very rapidly. The primary software challenges are twofold - support
for the HUBzero framework and challenges related to the software contributed by the community.

The group has learned that community contributions are maximized with a tolerant licensing approach. HUBzero uses an
LGPLv3 license so contributors can create unique components and license as they choose. If they make changes to source
code, the original license must be maintained for redistribution. As far as contributed resources, these must be open access,
but not necessarily open source. This allows contributors to meet the requirements of their institutions and funding agencies.
Quality is maintained via user ratings. Documentation is encouraged and nanoHUB supplies regression test capabilities, but
the user community provides ratings, poses questions and contributes to wishlists and citation counts - all of which incentivize
code authors.

The Terrel paper describes support for the Python scientific community through two major efforts - the SciPy conference and the
NumFOCUS foundation. Reliance on software in science has driven a huge demand for development, but this development is typically done as a side effort
and often in a rush to publish without documentation and testing.
While created by academics, software support often falls to industrial institutions. SciPy brings together industry, government,
and academics to share their code and experience in an open environment.

NumFOCUS promotes open, usable scientific software while sustaining the community. Specific activities include educational programs,
collaborative research tools and documentation and
promotion of high-level languages, reproducible scientific research, and open-code development.
Governance of the non-profit is a loose grantor-grantee relationship with projects allowing for monies to be placed in the groups accounts.
This has raised money to hire developers for open code development, maintain testing machines, organize the PyData conference series, and sponsor community members to attend conferences.

Software sustainability relies on contributions from all sectors of user community. Together SciPy and NumFOCUS support these sectors. By maximizing contributions growing the user base they help develop and mature Python.

The L\"{o}ffler paper describes the Cactus project which was started in 1996 by participants in USA Binary Black Hole Alliance
Challenge. Cactus has a flesh and thorns model - a community-oriented framework that allows researchers to easily work
together with reusable and extendable software elements. Modules are compiled into an executable and can remain dormant, becoming
active only when parameters and simulation data dictate. Users can  use modules written by others or can write their own
modules without changing other code. The community has grown and diversified beyond the original science areas.

The paper points out 4 keys to sustaining the community - modular design, growing a collaborative community, career paths and credit.
In modular design, the Cactus project went far beyond standard practices of APIs.
Domain specific languages (DSLs) allow decoupling of components - for example I/O, mesh refinement, PAPI counters, and
boundary conditions abstracted from science code. In academia, publications are the main currency of credit. Because the
project connects code developments to science, the work is
publishable and modules are citable. Because of the open source, modular approach, programmers can see the impact of their contributions and often continue
work after graduation. Career paths remain a challenge, however. Tasks that are essential from a software engineering perspective
are often not rewarded in academia. The best programmers in a science environment often have multidisciplinary expertise.
That also is not rewarded in academia.

The Wilkins-Diehr paper describes an NSF software institute effort to build a community of those creating science gateways or
web portals for science. These gateways, as described in some of the other papers in this section, can be quite capable, having a
remarkable impact on science.

This paper mentioned challenges highlighted by other papers in this section - mainly the conflict between funding for research
vs infrastructure and the challenges around getting academic credit for infrastructure. Because the authors have studied
many projects, they've also observed how development is often done in an isolated hobbyist environment. Developers are unable
to take advantage of similar work done by others. isolation even when projects have common needs.
But often projects struggle for sustainable funding because they provide infrastructure to conduct
research and many times only the research is funded. Gateways also may start as small group research project, taking off in popularity
once they go live without any long term plans for sustainability or without resources in the project to plan for such.
Subsequent disruptions in service can limit effectiveness and test the limits of the research community's trust. The impact
of gateways can be increased substantially if we understand what makes them successful.

Recommendations from an early study of successful gateways include the following. Leadership and management teams should
design governance to represent multiple strengths and perspectives, plan for change and turnover in the future,
recruit a development team that understands both the technical and domain-related issues, consider sustainability and measure success early and often.
Successful projects recognize the benefits and costs of hiring a team of professionals, demonstrate credibility through stability and
clarity of purpose, leverage the work of others and plan for flexibility.
Successful projects identify an existing community and understand that communities' needs before beginning. Projects adapt
as the communities' needs evolve. For funders, the lifecycle of technology projects must be considered. Solicitations should
be designed to reward effective planning, recognize the benefits and limitations of both technology innovation and reuse,
expect adjustments during the production process, copy effective models from other industries and sectors, and encourage partnerships that support gateway sustainability.

Through a business incubator type approach, the institute plans to provide a variety of services that could be shared amongst
projects. Consultation and resources on topics such as business plan development, software engineering practices,
software licensing options, usability, security and project management as well as a software repository and hosting
service will be available. Experts will be available for multi-month assignments to help research teams build their own gateways,
teaching them how to maintain and add to the work after the collaboration ends. Forums, symposia and an annual conference
will connect members of the development community. A modular, layered framework that supports community contributions and
allows developers to choose components will be delivered and finally
workforce development activities will help train the next generation for careers in this cross- disciplinary area and
build pools of institutional expertise that many projects can leverage. Shared services and forums can dramatically reduce the cost
of building and sustaining gateways. Workforce development can encourage technology professionals to remain in the sciences.


\subsection*{Papers}

\subsubsection*{Communities Part 1 Papers}

\begin{itemize}

\item Reagan Moore. Extensible Generic Data Management Software
  \cite{Moore_WSSSPE}

\item Karen Cranston, Todd Vision, Brian O'Meara, Hilmar Lapp. A
  grassroots approach to software sustainability
  \cite{Cranston_WSSSPE}

\item J.-L. Vay, C. G. R. Geddes, A. Koniges, A. Friedman,
  D. P. Grote, D. L. Bruhwiler. White Paper on DOE-HEP Accelerator
  Modeling Science Activities \cite{Vay_WSSSPE}

\item Ketan Maheshwari, David Kelly, Scott J. Krieder, Justin M. Wozniak, Daniel S. Katz, Mei Zhi-Gang, Mainak Mookherjee. Reusability in Science: From Initial User Engagement to Dissemination of Results \cite{Maheshwari_WSSSPE}

\item Edmund Hart, Carl Boettiger, Karthik Ram, Scott Chamberlain. rOpenSci -- a collaborative effort to develop R-based tools for facilitating Open Science \cite{Hart_WSSSPE}

\end{itemize}

\subsubsection*{Communities Part 2 Papers}

\begin{itemize}

\item L. Christopherson, R. Idaszak, S. Ahalt. Developing Scientific Software through the Open Community Engagement Process \cite{Christopherson_WSSSPE}

\item Marlon Pierce, Suresh Marru, Mark A. Miller, Amit Majumdar, Borries Demeler. Science Gateway Operational Sustainability: Adopting a Platform-as-a-Service Approach \cite{Pierce2_WSSSPE}

\item Lynn Zentner, Michael Zentner, Victoria Farnsworth, Michael
  McLennan, Krishna Madhavan, and Gerhard Klimeck, nanoHUB.org:
  Experiences and Challenges in Software Sustainability for a Large
  Scientific Community \cite{Zentner_WSSSPE}

\item Andy Terrel. Sustaining the Python Scientific Software Community
  \cite{Terrel_WSSSPE}

\item Frank L\"{o}ffler, Steven R. Brandt, Gabrielle Allen and Erik
  Schnetter. Cactus: Issues for Sustainable Simulation Software
  \cite{Loffler_WSSSPE}

\item Nancy Wilkins-Diehr, Katherine Lawrence, Linda Hayden, Marlon Pierce, Suresh Marru, Michael McLennan, Michael Zentner, Rion Dooley, Dan Stanzione. Science Gateways and the Importance of Sustainability \cite{Wilkins-Diehr_WSSSPE}

\end{itemize}

In addition, the following paper from another area was also discussed
in this area.

\begin{itemize}

\item Marcus Hanwell, Amitha Perera, Wes Turner, Patrick O'Leary,
  Katie Osterdahl, Bill Hoffman, Will Schroeder. Sustainable Software
  Ecosystems for Open Science \cite{Hanwell_WSSSPE}

\end{itemize}

\subsubsection*{Industry \& Economic Models}

\begin{itemize}

\item Anne C. Elster. Software for Science: Some Personal Reflections
  \cite{Elster_WSSSPE}

\item Ian Foster, Vas Vasiliadis, Steven Tuecke. Software as a Service
  as a path to software sustainability \cite{Foster_WSSSPE}

\item Marcus Hanwell, Amitha Perera, Wes Turner, Patrick O'Leary,
  Katie Osterdahl, Bill Hoffman, Will Schroeder. Sustainable Software
  Ecosystems for Open Science \cite{Hanwell_WSSSPE}

\end{itemize}

In addition, the following papers from other areas were also discussed
in this area.

\begin{itemize}

\item Brian Blanton, Chris Lenhardt, A User Perspective on Sustainable
  Scientific Software \cite{Blanton_WSSSPE}

\item Markus Blatt. DUNE as an Example of Sustainable Open Source
  Scientific Software Development \cite{Blatt_WSSSPE}

\item Dhabaleswar K. Panda, Karen Tomko, Karl Schulz, Amitava
  Majumdar. The MVAPICH Project: Evolution and Sustainability of an
  Open Source Production Quality MPI Library for HPC
  \cite{Panda_WSSSPE}

\item Andy Terrel. Sustaining the Python Scientific Software Community
  \cite{Terrel_WSSSPE}

\end{itemize}

\subsubsection*{Education \& Training}

\begin{itemize}

\item Ivan Girotto, Axel Kohlmeyer, David Grellscheid, Shawn
  T. Brown. Advanced Techniques for Scientific Programming and
  Collaborative Development of Open Source Software Packages at the
  International Centre for Theoretical Physics (ICTP)
  \cite{Girotto_WSSSPE}

\item Thomas Crawford. On the Development of Sustainable Software for
  Computational Chemistry \cite{Crawford_WSSSPE}

\end{itemize}

In addition, the following papers from other areas were also discussed
in this area.

\begin{itemize}

\item Charles R. Ferenbaugh, Experiments in Sustainable Software
  Practices for Future Architectures \cite{Ferenbaugh_WSSSPE}

\item David Koop, Juliana Freiere, Cl\'{a}udio T. Silva, Enabling
  Reproducible Science with VisTrails~\cite{Koop_WSSSPE}

\item Sean Ahern, Eric Brugger, Brad Whitlock, Jeremy S. Meredith,
  Kathleen Biagas, Mark C. Miller, Hank Childs, VisIt: Experiences
  with Sustainable Software \cite{Ahern_WSSSPE}

\item Sou-Cheng (Terrya) Choi. MINRES-QLP Pack and Reliable
  Reproducible Research via Staunch Scientific Software
  \cite{Choi_WSSSPE}

\item Frank L\"{o}ffler, Steven R. Brandt, Gabrielle Allen and Erik
  Schnetter. Cactus: Issues for Sustainable Simulation Software
  \cite{Loffler_WSSSPE}

\item Erik Trainer, Chalalai Chaihirunkarn, James Herbsleb. The Big
  Effects of Short-term Efforts: A Catalyst for Community Engagement
  in Scientific Software \cite{Trainer_WSSSPE}

\end{itemize}

\subsection{What are communities?}
\subsection{Challenges of community}
include metrics for community success (e.g., more external
contributors than internal).
\subsection{Admirable scientific software communities}
\begin{itemize}
\item example communities
\item their (quick) origin stories.
\end{itemize}
\subsection{Resources for learning about software communities}
\begin{itemize}
\item academic fields studying communities (and software communities)
\item courses on online communities
\item books
\item need for software carpentry module on organizing communities?
\end{itemize}

\section{Other Discussion} \label{sec:other}

\note{from the misc questions/answers/observations part of the google doc?}

\note{all to put notes here of stuff that doesn't fit elsewhere, if any}

\todo{perhaps remove this section}


\section{Cross-cutting Issue: Defining Sustainability \note{lead: Daniel S. Katz}}  \label{sec:cross-cutting}

The question of what was meant by ``sustainability'' in the context of software
came up in many different parts of the workshop, specifically in the first
keynote (\S\ref{sec:keynote1}), the Developing and Maintaining Software panel, and the Policy panel.

In the opening keynote, Philip Bourne suggested that perhaps sustainability can be defined
as the effort that happens to make the essential things continue. This leads to
having to decide what it is that we want to sustain, whether what we want to sustain is
valuable, and finally, who would care if it went away, and how one measures
how much they care.

In the Developing and Maintaining Software panel, there was some discussion
on this topic: what does sustainability mean? It was pointed out that
OSS Watch proposes a Software Sustainability Maturity Model to address
the issue of how sustainability a particular element of software is, and says
that this sustainability is important. ``When choosing
software for procurement or development reuse - regardless of the
license and development model you will use - you need to consider
the future. While a software product may satisfy today's needs, will
it satisfy tomorrow's needs? Will the supplier still be around in
five years' time? Will the supplier still care for all its customers
in five years' time? Will the supplier be responsive to bug reports
and feature requests? In other words, is the software sustainable?''~\cite{OSS-ssmm-web}

Attendees suggested that a key question that the definition of sustainability is
one issue on which the community needs to agree, and that ideally, an initial
definition would be determined during the workshop, or at least some progress
would be made towards this goal.  Another topic that came up is what the
goal of sustainability is.  Perhaps it is reproducible science, or
persistence, or quality, or something else.  Similarly, some attendees want
to understand how success in sustainability is measured.  How does a group of
developers know when they have actually achieved sustainable software?
This led to a comment that sustainability should be addressed throughout the
full software life cycle.

Another topic that came up during the Developing and Maintaining Software
panel is the relationship of sustainability to other software attributes.  Attendees
asked ``what is the relationship between sustainability and provenance?'' And,
``is usability separate from sustainability or a fundamental part of it?''

In addition, the Policy panel had a large amount of discussion about
defining sustainability, as one of the subtopics in that panel was
Modeling Sustainability, and modeling requires defining what will be modeled.

In this subtopic, two papers included discussion of the definition of
sustainability.  Venters et al.~\cite{Venters_WSSSPE} mentioned
that this is a rather ambiguous concept, and that the lack of an accepted
definition gets in the way of integrating the concept into software engineering.
They suggest that sustainability is a non-functional requirement, and that the
quality of software architectures determines sustainability.  They then
propose that sustainability is a measure of a set of central quality attributes:
extensibility, interoperability, maintainability, portability, reusability, and scalability.
Finally, they develop an architecture evaluation framework based on scenarios
that help to illuminate how to measure quality or sustainability at
the architectural level.

Katz and Proctor~\cite{Katz_WSSSPE} included a set of questions that could
be used to measure software sustainability, and though these questions might falsely lead
to yes or no answers, the complete set would determine a range of values for
sustainability. These questions are:
\begin{itemize}
\item Will the software continue to provide the same functionality in the future, 
      even if the environment (other parts of the infrastructure) changes?
\item Is the functionality and usability clearly explained to new users? 
\item Do users have a mechanism to ask questions and to learn about the element?
\item Will the functionality be correct in the future, even if the environment changes?
\item Does it provide the functionality that existing and future users want?
\item Does it incorporate new science/theory/tools as they develop?
\end{itemize}

Lenhardt~\cite{lenhardt-wssspe1-panel} summarized the contributions of the Modeling
Sustainability papers in the panel.
As shown in Table~\ref{tab:defining-sustainability}, for each paper he
discussed what software meant, whether there was a definition of sustainability, and
what the approach was to either understand or evaluate sustainability.

\begin{table}[t]
  \begin{scriptsize}
    \begin{center}
      \caption{Summary of Modeling Sustainability papers from Policy Panel.  Adapted from \cite{lenhardt-wssspe1-panel}.}
      \label{tab:defining-sustainability}
      \begin{tabular}{|p{2.3cm}|p{3.6cm}|p{4.4cm}|p{4.8cm}|}
                \hline
{\bf Paper/Authors}
& {\bf Software}
& {\bf Sustainability}
& {\bf Approach to Understand or Evaluate Sustainability} \\
                \hline
Calero, et al. \cite{Calero_WSSSPE}
& General notion of software. Not explicitly defined.
& Sustainability is linked to quality.
& Add to ISO \\
                \hline
Venters, et al. \cite{Venters_WSSSPE}
& Software as science software; increasingly complex; service-oriented computing
& Extensibility, interoperability, maintainability, portability, reusability, scalability, efficiency
& Use various architecture evaluation approaches to assess sustainability \\
                \hline
Pierce, et al. \cite{Pierce_WSSSPE}
& Cyberinfrastructure software
& Sustainable to the extent to which there is a community to support it
& Open community governance \\
                \hline
Katz and Proctor \cite{Katz_WSSSPE}
& E-research infrastructures (i.e. cyberinfrastructure)
& Persisting over time, meeting original needs and projected needs
& Equates models for the creation of software with sustaining software \\
                \hline
Lenhardt, et al. \cite{Lenhardt_WSSSPE}
& Broadly defined as software supporting science
& Re-use; reproducible science
& Comparing data management life cycle to software development life cycle \\
                \hline
Weber, et al. \cite{Weber_WSSSPE}
& Software broadly defined; a software ecosystem
& Software niches
& Ecological analysis and ecosystem \\
                \hline
     \end{tabular}
    \end{center}
  \end{scriptsize}
\end{table}


Finally, in the workshop's closing session, one of the discussion topics was
what success would look like for the set of WSSSPE activities beyond just the workshop.
One of the answers that was suggested was
having an agreed version of what we mean by sustainability.


\section{Case Studies \note{lead: Ketan Maheshwari, contributors: Andy Terrel}} \label{sec:use-cases}


For the purpose of studying software case studies, we classify the software
projects discussed in the workshop in three broad categories. The first is 
`utility software', software tools that enable and/or facilitate the
development of other tools and techniques to carry out scientific work. The second is
`scientific software', software that was developed with an aim to solve a
class of scientific problems. The third is `infrastructure software', software
developed to efficiently utilize new research infrastructure.

These categories have some overlap, and in some cases, the
case studies do not cleanly fit into a single category.
Our classification is based on two criteria: first, the character
of a software project that dominates and where it `seems' to be the best fit, and, second,
the the workshop discussion session where the paper about
the project was discussed. 

%and point it out in our
%discussion. 

Our discussion of the case studies is aimed at understanding the practicalities of
the many points discussed in the workshop, and we hope to extract the best
practices and associate them with the projects. We discuss how the cases are
applicable to and have impacted user communities over their lifetime, isolating
similarities and differences between projects. We try to define the takeaways for a new
project and the lessons learned. We try to understand how the course of current and
future development projects might be altered to implement lessons learned. 

\subsection{Utility Software}
Utility software potentially has the broadest community impact of the three
categories. This is especially relevant in the modern era of social computing.
In the context of this paper, a utility software project could be defined as
general purpose software applicable to one or many generic tasks and/or
enabling other software to run by providing a suitable environment. Examples
are collaborative development framework such as \emph{GitHub} and
\emph{Bitbucket}, workflow and scripting frameworks such as \emph{Galaxy},
\emph{Swift}, \emph{Globus} and \emph{VisTrails}, and visualization frameworks
such as \emph{visIT}. Often projects start with specific, small set of
requirements and users in mind but expand to provide more generic functionality
and appeal to a wider user community. Owing to its generic nature, utility
software finds a wider users communities which in turn leads to a higher
participation in development. Consequently, the process becomes user-driven and
self sustaining. Software developed in such scenario often benefits from
community driven development, deployment and maintenance best practices. For
instance the \emph{Galaxy} project has significantly benefitted from user
community participation in its development. It follows agile software
development practices and implements standard practices such as test-driven
development and socialized bug managing practices via \emph{trello}. Galaxy
\emph{histories} and \emph{toolshed} offers easy community sharing of data and
tools further promoting a collaborative environment. The project very closely
follows the guidelines described in~\cite{Carver_WSSSPE} and many
from~\cite{Prlic_WSSSPE}. 

Another paradigm in utility software is the software delivered as service over
the web. With increasing popularity of cloud-based computational environments,
many users are leaning towards tools used as service. \emph{GitHub} can be
arguably considered one such tool, catering to collaborative development. For
scientific users \emph{Globus} based tools are case of service based utility
software discussed during the workshop. The data movement, authentication and
sharing services offered by Globus can be easily used over the web by
collaborating researcher.

\subsection{Scientific software}
Scientific software consists primarily of special purpose software that was developed for a target use-case
scenario or keeping some fixed/frozen requirements in mind. Software projects
pertaining to specific scientific domain often tend to be in a niche and the
user community tends to be small to medium. Numerical accuracy and algorithmic
optimization are some of the paramount requirements of most scientific
software. Advantages of a smaller codebase and requirements include stability,
ease of installation, configurability and integration into larger systems.
While the software can stay stable and require relatively low maintenance, the
responsibility is often on the shoulders of a very few developers. Community
participation in the development tends to be low. Development is linear and
simplistic with a limited scope to follow software best practices. Domain
science oriented software developed to solve  specific problems, e.g.,
\emph{DUNE}, \emph{PETSc}, \emph{MINRES-QLP}, \emph{FASTMath}. Sustainability
of such software is often a significant challenge. Many submissions reported the
software considered a `byproduct' of the actual research.  Others contended that
the software was not the main funded part of their research. One exception was
\emph{Kitware}, which while being a software product specializing in scientific
process, has a core focus of developing communities around software processes.
One example of this process is the development of the \emph{CMake} build
utility, which started out as a building tool for \emph{ITK} but grew to become
a generic build utility for C++ projects. 

\subsection{Infrastructure software}
%TODO: expand
Infrastructure software is developed to better utilize particular, often new, infrastructures and architectures,
e.g., \emph{MVAPICH}, \emph{visIT}. Explores uncharted territories. Highest
risk since the projects are experimental. \note{Dan: so is the only difference here that
this category is aimed at new systems, where utility software is not?  If so, perhaps this should be incorporated into that subsection, so there would just be two subsections. Something in the framing of this subsection doesn't feel right to me.} Successful projects reap high rewards
and have longer usage span. By nature, they overlap with the utility software
as they must serve as a facilitator to run ordinary codes. Long
requirement-development-test cycles.  Wide community support but tends to be
handled by specialists rather then end users. Promotes collaborations across
the breadth (e.g., different departments) and depth (e.g., stakeholders within a
department) of community, one of the key ingredient of a sustainable process.
Ferenbaugh~\cite{Ferenbaugh_WSSSPE} discusses the risks and challenges
associated with software projects dealing with new architectures. To leverage
the power of accelerators such as GPUs and MICs new code and libraries are
required. The experience of efforts as described in~\cite{Ferenbaugh_WSSSPE}
met with a limited success but nonetheless with many invaluable lessons were
learned about influential cultural and technical aspects in sustainable software
development practices.

Summary of above discussed cases: lessons, trends, commonalities, is there an
ideal sustainability scenario?

\note{Dan: I think there's some overlap
  between this and the discussion in the Developing and Support
  Software section.  Perhaps this should go in the cross-cutting
  issues section?  Let's write it down here - then we can decide if it goes elsewhere}
\note{Ketan: yes, may be this section will split and merge into other sections}

\section{Conclusions} \label{sec:conclusions}

\todo{pull the discussion together}

\todo{add some analysis of the deficiencies and difficulties that are
  present in different fields, and those that are common?}

\todo{say something about licensing - lessons, advice, etc.?}

\subsection{Recommendations or Lessons}

\note{if needed.}

\subsection{Follow up actions}

\note{or at least the discussion about them, and the current plans for
  future events.}

\note{conclusions from pre-workshop paper follows}

The WSSSPE workshop has begun an experiment in how we can
collaboratively build a workshop agenda. However, contributors also
want to get credit for their participation in the process. And the
workshop organizers want to make sure that the workshop content and
their efforts are recorded.  Ideally, there would be a service that
would be able to index the contributions to the workshop, serving the
authors, the organizers, and the larger community. But since there
isn't such a service today, the workshop organizers are writing this
initial report and making use of arXiv as a partial solution to
provide a record of the workshop.

After the workshop, one or more additional papers will be created that
will include the discussions at the workshop. These papers will likely
have many authors, and may be submitted to peer-reviewed journals.


\section*{Acknowledgments}

\todo{anyone who needs to put something in here should}

Some of the work by Katz was
supported by the National Science Foundation while working at the
Foundation; any opinion, finding, and conclusions or recommendations
expressed in this material are those of the author(s) and do not
necessarily reflect the views of the National Science Foundation.


\appendix
\section{List of attendees \note{lead: Shel Swenson}}


The following is a partial list of attendees who were recorded on the
Google doc~\cite{WSSSPE1-google-notes} that was being used for live note taking at the workshop, or by the SC13 student volunteers, with some additions also made by the authors of this report.


\begin{multicols}{3}
\setlength{\parindent}{0pt}
%in theory, should save the old value then set it back after this section, but since this is the end...

Jay Alameda

Gabrielle Allen

David Andrs

Brian Austin

Lorena A Barba

David Bernholdt

Phil Bourne

Karl Broman

Sharon Broude Geva

Jed Brown

Maxine Brown

David Bruhwiler

Bruno Bzeznik

Alexandru Calotoiu

Jeffrey Carver

Shreyas Cholia

Peng Huat Chua

Neil Chue Hong

John W. Cobb

Timothy Cockerill

Karen Cranston

Rion Dooley

Anshu Dubey

Marat Dukhan

Ian Foster

Juliana Freire

Jeffrey Frey

Derek Gaston

AllisonGehrke

Brian Glendenning

Christian Godenschwager

Derek Groen

Edmund Hart

Magne Haveraaen

Steven Heaton

Oscar Hernandez

James Hetherington

James Hetherington

Simon Hettrick

Jonathan Hicks

Kenneth Hoste

James Howison

Daisie Huang

Shao-Ching Huang

Tsutomu Ikegami

Kaxuya Ishimura

Christian Iwainsky

Craig Jackson

Wesley Jones

Randall Judd

Shusuke Kasamatsu

Daniel S. Katz

Kerk Kee

Kellie Kercher

Mads Kristensen

Carolyn Lauzon

Arnaud Legrand

Chris Lenhardt

Michael Levy

Frank L\"{o}ffler

Monica Luecke

Simon A. F. Lund

Arthur Maccabe

Paul Madden

Louis Maddox

Philip Maechling

Ketan Maheshwari

Brian Marker

Suresh Marru

Cezary Mazurek

James McClure

Matt McKenzie

Chris Mentzel

Paul Messina

Mike Mikailov

J. Yates Monteith

Reagan More

Rafael Morizawa

Pierre Neyron

Lucas Nussbaum

Patrick O'Leary

Manish Parashar

Cody Permann

Jack Perdue

John Peterson

Quan Pham

Marlon Pierce

Heather Piwowar

David Proctor

Sara Rambacher

Nicolas Renon

Jason Riedy

Todd Rimer

Bill Sacks

Andreas Schreiber

William Scullin

Andrew Slaughter

Jaraslaw Slawinski

Arfon Smith

Spencer Smith

James Spencer

Eric Stahlberg

Timothy Stitt

Hyoshin Sung

Fr\'{e}d\'{e}ric Suter

Shel Swenson

Yoshio Tanaka

Andy Terrel

George  Thiruvathukal

Keren Tomko

John Towns

Erik Trainer

Satori Tsuzuki

Matthew Turk

Eric Van Wyk

Colin C. Venters

Brice Videau

Tajendra Vir Singh

Von Welch

Nancy Wilkins-Diehr

Theresa Windus

Felix Wolf

Rich Wolski

Lynn Zentner

\end{multicols}



\bibliographystyle{plain}

\bibliography{wssspe}
\end{document}
